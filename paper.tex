%% The first command in your LaTeX source must be the \documentclass command.
%%%% Small single column format, used for CIE, CSUR, DTRAP, JACM, JDIQ, JEA, JERIC, JETC, PACMCGIT, TAAS, TACCESS, TACO, TALG, TALLIP (formerly TALIP), TCPS, TDSCI, TEAC, TECS, TELO, THRI, TIIS, TIOT, TISSEC, TIST, TKDD, TMIS, TOCE, TOCHI, TOCL, TOCS, TOCT, TODAES, TODS, TOIS, TOIT, TOMACS, TOMM (formerly TOMCCAP), TOMPECS, TOMS, TOPC, TOPLAS, TOPS, TOS, TOSEM, TOSN, TQC, TRETS, TSAS, TSC, TSLP, TWEB.
% \documentclass[acmsmall]{acmart}

%%%% Large single column format, used for IMWUT, JOCCH, PACMPL, POMACS, TAP, PACMHCI
\documentclass[acmlarge,screen]{acmart}

%%%% Large double column format, used for TOG
%\documentclass[acmtog, authorversion]{acmart}

%%%% Generic manuscript mode, required for submission
%%%% and peer review
%\documentclass[manuscript,screen,review, acmconf]{acmart}
%\documentclass[sigconf,anonymous]{acmart}

\usepackage{svg}
\usepackage{csquotes}

\newcommand{\NGO}{\textit{NGO}}
\newcommand{\NGOOne}{NGO1}
\newcommand{\PC}{\textit{GigCo}}
\newcommand{\PCOne}{GigCo1}
\newcommand{\PCTwo}{GigCo2}

%% Rights management information.  This information is sent to you
%% when you complete the rights form.  These commands have SAMPLE
%% values in them; it is your responsibility as an author to replace
%% the commands and values with those provided to you when you
%% complete the rights form.
\setcopyright{acmcopyright}
\copyrightyear{2018}
\acmYear{2018}
\acmDOI{10.1145/1122445.1122456}

%% These commands are for a PROCEEDINGS abstract or paper.
\acmConference[CHI'22]{CHI 2022}{April 30--May 06, 2022}{New Orleans, LA}
\acmBooktitle{ACM CHI 2022 Conference on Human Factors in Computing Systems,
  April 30--May 06, 2022, New Orleans, LA}
\acmPrice{15.00}
\acmISBN{978-1-4503-XXXX-X/18/06}


%%
%% Submission ID.
%% Use this when submitting an article to a sponsored event. You'll
%% receive a unique submission ID from the organizers
%% of the event, and this ID should be used as the parameter to this command.
%%\acmSubmissionID{123-A56-BU3}

%%
%% The majority of ACM publications use numbered citations and
%% references.  The command \citestyle{authoryear} switches to the
%% "author year" style.
%%
%% If you are preparing content for an event
%% sponsored by ACM SIGGRAPH, you must use the "author year" style of
%% citations and references.
%% Uncommenting
%% the next command will enable that style.
%%\citestyle{acmauthoryear}

%%
%% end of the preamble, start of the body of the document source.
\begin{document}

%%
%% The "title" command has an optional parameter,
%% allowing the author to define a "short title" to be used in page headers.
\title{Critically Engaging with Embedded Values through Constrained Technology Design}

%%
%% The "author" command and its associated commands are used to define
%% the authors and their affiliations.
%% Of note is the shared affiliation of the first two authors, and the
%% "authornote" and "authornotemark" commands
%% used to denote shared contribution to the research.
\author{Anonymous for review}
\affiliation{%
  \institution{Anon}
  \streetaddress{Anon}
  \city{Anon}
  \country{Anon}}
\email{Anon@Anon.org}

\author{Anonymous for review}
\affiliation{%
  \institution{Anon}
  \streetaddress{Anon}
  \city{Anon}
  \country{Anon}}
\email{Anon@Anon.org}

\author{Anonymous for review}
\affiliation{%
  \institution{Anon}
  \streetaddress{Anon}
  \city{Anon}
  \country{Anon}}
\email{Anon@Anon.org}

\author{Anonymous for review}
\affiliation{%
  \institution{Anon}
  \streetaddress{Anon}
  \city{Anon}
  \country{Anon}}
\email{Anon@Anon.org}

%%
%% By default, the full list of authors will be used in the page
%% headers. Often, this list is too long, and will overlap
%% other information printed in the page headers. This command allows
%% the author to define a more concise list
%% of authors' names for this purpose.
\renewcommand{\shortauthors}{Anon et al.}

%%
%% The abstract is a short summary of the work to be presented in the
%% article.
\begin{abstract}
The rise of gig-economy platforms has highlighted the impact platform and algorithm design can have upon workers’ experiences. This paper reports on an extended series of collaborative design engagements with a private company and an international NGO during the production of an IVR component for a gig-economy platform. We present the findings of a design ethnography undertaken during this process, and discuss how design decisions reflect how each party’s values, motivations and assumptions are embedded within the final technology. We argue that there exists a need for simple methods to assist designers and engineers to surface and critically engage with the potentially disparate values, priorities and assumptions held by themselves and system stakeholders. Based off of our findings, we contribute suggestions for how this can be done during the production of real-world systems through the application of constrained design as a values lever.
\end{abstract}

%%
%% The code below is generated by the tool at http://dl.acm.org/ccs.cfm.
%% Please copy and paste the code instead of the example below.
%%
\begin{CCSXML}
<ccs2012>
 <concept>
  <concept_id>10010520.10010553.10010562</concept_id>
  <concept_desc>Computer systems organization~Embedded systems</concept_desc>
  <concept_significance>500</concept_significance>
 </concept>
 <concept>
  <concept_id>10010520.10010575.10010755</concept_id>
  <concept_desc>Computer systems organization~Redundancy</concept_desc>
  <concept_significance>300</concept_significance>
 </concept>
 <concept>
  <concept_id>10010520.10010553.10010554</concept_id>
  <concept_desc>Computer systems organization~Robotics</concept_desc>
  <concept_significance>100</concept_significance>
 </concept>
 <concept>
  <concept_id>10003033.10003083.10003095</concept_id>
  <concept_desc>Networks~Network reliability</concept_desc>
  <concept_significance>100</concept_significance>
 </concept>
</ccs2012>
\end{CCSXML}

\ccsdesc[500]{Computer systems organization~Embedded systems}
\ccsdesc[300]{Computer systems organization~Redundancy}
\ccsdesc{Computer systems organization~Robotics}
\ccsdesc[100]{Networks~Network reliability}

%%
%% Keywords. The author(s) should pick words that accurately describe
%% the work being presented. Separate the keywords with commas.
\keywords{design ethnography, digital civics}


%%
%% This command processes the author and affiliation and title
%% information and builds the first part of the formatted document.
\maketitle

\section{Introduction}

The use of online platforms to find small jobs has seen a worldwide explosion in popularity in recent years \cite{Taylor2017, islam2019, Wood2019}. Dubbed the `gig-economy', it exists as a labour-based subset of the `sharing economy', where workers capitalise on their skills and spare time as underused assets for income generation, often in addition to other commitments \cite{Balaram2017}. The pace at which the gig-economy has grown and `disrupted' markets has presented its own issues, however, and employment law has struggled to keep up \cite{Minter2017}, resulting in fears of worker exploitation due to a lack of necessary protections and safety nets \cite{Balaram2017}. However, any laws which do get introduced also need to protect the flexibility and autonomy offered to workers by these platforms, which made them appealing to so many in the first place \cite{Wood2019}. Similarly, the technologies which power the gig-economy need to account and balance the disparate needs of two primary stakeholder groups: their customers (and, in turn, company profit), and their workers (whose requirements may vary, based on context and field of work \cite{carlos2021}). An understanding of how designers' assumptions and priorities are embedded into these systems is therefore essential, as such design decisions have the potential to both benefit and hinder these stakeholders.

While waiting for this official legislation and regulation of the gig-economy to be introduced, there have been calls for non-state actors---such as unions and NGOs---to negotiate to formal agreements with gig-economy companies, providing short-term benefits to workers and highlighting issues with existing legislation \cite{Minter2017}. This paper engages within this context, and presents findings from an extended series of engagements from collaborative design engagements with an international advocacy group (\NGO{}) and a start-up company (\PC{}) which runs a gig-economy platform for disadvantaged domestic workers in Dhaka, Bangladesh. The product of these engagements is an Interactive Voice Response (IVR) system, designed to connect the predominantly offline domestic workers to \PC{}'s digital infrastructure in order to distribute and manage jobs. We present the results of a design ethnography undertaken during this process, discussing the system's design and how both party's values, motivations and assumptions have been embedded within it. We contribute our insights from this series of design engagements, suggestions for how designers and engineers can critically engage with their own values as a part of the design process, and discuss the potential pitfalls non-state actors face when attempting to formally advocate for workers while working within a framework of disempowerment.
\section{Related Work}

\subsection{Surfacing Values in Technology Design}

The critical analysis of stakeholders' human values to inform technology design and use has become an increasingly popular research subject within HCI \cite{shilton2018}, with contexts spanning from social media \cite{DeVito2021}, to education \cite{richardson2017} and games \cite{flanagan2014}. Traditionally, algorithms, platforms and databases have been discussed as if they are ethically neutral, and as a result values-based inquiry has been left outside of the scope of developers' design practice \cite{Shilton2013}. As the designs of systems and infrastructures can have implicit or explicit political qualities embedded within them \cite{winner1980}, the ethnographic study of such infrastructures can provide insights into designers' decision-making processes \cite{Star1999}. One of the most influential frameworks for approaching value-conscious design is Value Sensitive Design (VSD): an adaptable, tripartite methodology that uses theoretical, empirical, and technical approaches in combination with a guiding list of suggested values to scaffold philosophical analyses of systems or design spaces \cite{friedman2006}. VSD has been previously reconfigured to focus on empirical techniques, such as photo elicitation and defamiliarization, which can assist in the discovery of non-prescribed values \cite{LeDantec2009}. Such probes have been used to aid in the discovery of stakeholders' goals, priorities, preferences and expectations \cite{flanagan2014, GrowAGame}, or to support explicit and implicit discussions of participant values \cite{Alshehri2020}. Other approaches, such as socio-technical integration, utilise research methods such as semi-structured interviews with designers to promote ethical reflection upon design decisions \cite{fisher2007}. 

However, such values-oriented processes are not commonly deployed by small and medium-sized software development teams in commercial settings, given their reliance on specialist research techniques and a common perception of them being slow or unnecessary \cite{Shilton2013}. Within these environments, designers often act as advocates pushing for human-centred design, often espousing values in tension with their organization's own interests \cite{Chivukula2020}. Shilton argues that having a `values advocate' within an organisation can be a viable approach, but acknowledges arguments that having a single advocate risks the marginalization of other voices, and that the need for such roles can be difficult to justify to leadership \cite{shilton2018, manders2013, Borning2012}. While such approaches are effective ways of critically engaging with stakeholders' values \cite{DeVito2021}, the required time investment and knowledge of literature and methodology (e.g. designing probes, conducting interviews) effectively requires the presence of a researcher---an unreasonable expectation of smaller companies, who are often unlikely to even have dedicated UX designers due to the typical prioritisation of functionality over form, usability, and even ethics \cite{Ardito2014, Shilton2013}.

Shilton recommends the use of `values levers': informal practices that call attention to designed infrastructure, serving as effective entry points for value discussions during the process of technology development \cite{Shilton2013, shilton2018}. Whereas VSD provides grounded reasoning through theoretical, empirical and technical investigations for values-based design decisions, values levers instead act as provocations: prompting moments of reflection and (sometimes emotional) reaction rather than evidence-based reasoning with scientific rigour. While resulting arguments for design rationale may not hold up to academic scrutiny, this is likely to be a non-issue for many within commercial contexts, and is arguably counter-balanced by values levers requiring significantly less work. Examples of such levers include: designers and engineers self-testing their creations and encountering discomfort at how their data is used \cite{Shilton2013}; designers explaining their decision-making across disciplinary barriers, requiring re-framing and re-examination of practices \cite{shilton2018}; and online communities of mobile app developers reflecting upon why technical constraints are put in place by mobile platform holders \cite{shilton2019}. Such examples suggest that values levers can be introduced as effective prompts for reflection within a commercial development environment, without the need for interventions by specialist researchers. However, Shilton argues that in order for values levers to exist, they have to be deployed by practices and agents \cite{Shilton2013}. Outside of the examples given, little research has explored what practices could be used to purposefully deploy values levers within a commercial context to promote the discussion of values and practitioner reflection.

\subsection{The Gig Economy \& South Asia}

To lower costs and support scalability, gig economy platforms typically use algorithmic approaches to automate the allocation of the `most suitable' workers to jobs \cite{Wood2019}. In theory, this allows gig work platforms to offer high levels of flexibility and autonomy, as workers can ostensibly have control over when and where they work \cite{Wood2019, Balaram2017, carlos2021}. The success of ride-sharing and delivery services such as Uber has led to the `gig' model being explored in many other service industries \cite{Balaram2017}. However, the recent popularity of gig work has prompted both caution and criticism: the industry's focus on quantified worker ratings and algorithmic assignment has been shown to result in low pay, social isolation, and overwork \cite{Wood2019}. Questions have also been raised around whether current employment laws are capable of addressing workers' needs, such as sickness protection, when applied outside of traditional employment models: adding to fears of exploitation through a `race to the bottom' of cheap pricing and low-cost labour \cite{Taylor2017,Balaram2017}. 

Nevertheless, gig work has remained popular, particularly in South Asia: India and Bangladesh are amongst the fastest growing freelancer markets in the world \cite{Payoneer2019}. Ride sharing services are increasingly popular within Bangladesh \cite{islam2019} and their unregulated growth has encouraged tens of thousands of rural and suburban youths to migrate to metropolitan areas, prompting calls for regulatory action \cite{Fairwork2021}. Ahmed argues that those in Bangladesh without access to digital technologies cannot access these new employment opportunities \cite{Ahmed2020}, further deepening an existing digital divide and highlighting concerns around workers being underpaid, overworked, and constantly monitored \cite{Irani2013}. Minter argues that, while there is a need for governments to introduce enforceable labour standards, this will take time: suggesting that while government solutions are being negotiated, non-state actors should work with gig economy companies on formal agreements to support workers' fair treatment and identify current issues \cite{Minter2017}.


\subsection{HCI \& the Gig Economy }

The HCI research community's involvement in the gig economy can be traced back to its undiluted form, `crowdworking', where online workers are given microtasks and paid per acceptable completion. Such platforms have been previously used as cheap and easily accessible sources of research participants (e.g. \cite{mason2012conducting, mcnaney2016, Othman2017}). While early academic discourse frequently focused on improving such platforms' efficiencies, comparatively little investigation was taken into the workers themselves: their socio-economic status and the impact of the platforms' designs on them \cite{Jacques2019, Irani2013}.

Recent years have seen more critical, values-based analyses: Martin et al. note the dehumanising rhetoric surrounding crowd workers (e.g. `\textit{artificial artificial intelligence}', and `\textit{cogs in the machine}'), and how such terminology makes them easier to regard as `troublesome components' to be controlled, rather than real human stakeholders worthy of design considerations \cite{martin2016}. A common frustration relating to gig economy platforms is a lack of transparency: the deeper functioning of such systems (e.g. the specifics of work assignment algorithms) is often opaque to the worker, leading to worker frustration and a balance of power in favour of clients and platform holders \cite{martin2016}. Furthermore, gig economy platforms are generally not designed to support communications between workers, which has been identified as one factor limiting gig worker collective bargaining \cite{Hara2018}. This weak bargaining power leads to the pace of work being determined by direct demands from clients, heightened by a lack of job security and a frequent oversupply of labour \cite{Wood2019}. Lee et al. argue that increased transparency in the assignment process could elicit greater cooperation with work assignments, especially undesirable ones: because the current supply-demand control algorithms do not account for human factors (such as workers' capability and motivations), their use in motivating and controlling human behaviors created distrust of the system in workers \cite{lee2015}. Lee also noted that the practical opacity of a platform's algorithm can lead to workers resorting to speculation and sensemaking through external channels, such as with other workers on social media platforms \cite{lee2015}. Raval \& Dourish note that gig platforms' monetisation models frequently erase the distinction of work and `related work' (such as care labour), and that platforms place workers' own bodies and possessions as the sites of engagement between clients and corporations: placing additional focus on workers' emotional performance, bodily presence and timeliness \cite{raval2016}. This combination of opaque, quantified evaluation and an apparent accountability for every interaction creates a hyper-awareness of clients' ratings and the potential for psychological stress \cite{lee2015}.

Introducing technology interventions without care runs the risk of amplifying existing inequalities amongst workers, or even creating new power dynamics within a given platform's economy \cite{martin2016}. In response to these issues, Alvarez et al. call for a greater worker-centred perspective in the design of gig economy platforms, focusing on transparency, professional development, networking and an avoidance of power asymmetry \cite{carlos2021}. Designing the algorithms used by gig economy platforms to manage and assign workers in a human-centred approach will require practical methods of identifying the values and requirements of all stakeholders, including those of the designers and engineers \cite{lee2015}.
\section{Context}

During this project we (members of an HCI research group based in Australia) entered into a three-way partnership with the Bangladeshi branch of an international NGO (anonymised as `\NGO{}') and \PC{} (anonymised), a gig-economy startup company based in Dhaka, Bangladesh (BD). Full ethical approval was received from our institutional review board before work commenced.

\NGO{} was in the midst of a project aiming to improve the well-being of female domestic workers in BD, who are typically hired to cook, clean, do laundry or even care for the children of a household. Estimates range between there being 2 million \cite{DWRN2011} and 4 million \cite{Ashraf2019} domestic workers in BD, with around 80\% being women or girls \cite{Ashraf2019}. Despite their prevalence, domestic work in BD was only recognised as an informal profession in 2015, with the introduction of the Domestic Workers Protection and Welfare Policy (DWPWP) \cite{Islam2017}. Prior to this, domestic workers were not entitled to time off, were not legally assured `fair' wages (85\% live under the poverty line \cite{BILS2015}), and there were few legal protections from abuse and harassment within their places of work \cite{IDWF2015}. However, adherence to these new policies has been inconsistent \cite{islam2016}, with a perceived lack of policy enforcement and reports of abuse still frequent \cite{DailyStar2018}. Within this context, \NGO{} aims to provide women domestic workers with skills training for formal job opportunities, to increase their awareness of their rights, and to support the BD government's capacity to enforce and monitor the implementation of the DWPWP. 

As discussed, gig-economy platforms are increasingly common in BD \cite{Ahmed2020}. In-line with the prevalent trend of technology-focused social interventions within BD (e.g. \cite{hasnayen2016, Faroqi2019}), \NGO{} chose this as a sector through which to promote their agenda and partnered with the gig-economy startup \PC{}. Aiming to be `the Uber of domestic workers', \PC{} runs an app-based service through which customers can request a domestic worker using their smartphone. In exchange for configuring their business towards the promotion of the DWPWP and the empowerment of domestic workers, \NGO{} are assisting \PC{} with the recruitment and training of several thousand domestic workers to support higher rates of pay. On \PC{}'s app, the `About Us' page notes that they `\textit{dream to build an ecosystem where every family employs trained, skilled and verified domestic helpers}' and that they aim to provide a `\textit{secure workplace for millions of domestic helpers through our platform and establish "domestic work" as a dignified profession}'. As such, \PC{}'s priority was to create a platform which would enable them to: i) meet commercial interests, and ii) empower domestic workers.

\PC{}'s employed domestic workers, like many women in BD, live in a digital divide driven by gender and socio-economic status \cite{Genilo2015}. Most lack access to smartphones, usually only having a `feature phone' which can place calls and send/receive SMS (although many have limited textual literacy, rendering SMS of limited value). As a result, the majority of \PC{}'s workforce are unable to directly interact with the digital infrastructure to respond to client requests. In response, \PC{} initially introduced `local guides': another tier of gig-economy workers who would connect customers to workers by taking orders from clients through the PC app and forwarding their details to domestic workers through phone calls. However, after several months \PC{} reflected that this process was costly and biased: reporting that local guides would often select workers based on personal preferences, leaving the process open to misuse. 

In late 2020, \PC{} and \NGO{} decided to explore alternative, automated solutions to bridge the domestic workers with the digital infrastructure. They decided to create an Interactive Voice Response (IVR) system: an automated telephone-based system which can be interacted with by callers through button presses. An automated IVR system was deemed to be more efficient and objective, as it provided a simple way to connect clients to users through an automated decision-making process that was thought to be less influenced by human bias. As a research team with experience of designing IVR systems and a pre-existing relationship with \NGO{}, we offered consultation and advice during the design process of this new system - whilst simultaneously carrying out a design ethnography of the process to reflect on how the values of the three parties were reflected in the design of the IVR system. 
\section{Methodology}

This study took place over the course of 11 months, performed remotely due to the global pandemic. During this time, the research team hosted 12 Zoom meetings with members of \PC{} (\PCOne{} \& \PCTwo{}), with each meeting running for an average of one hour. At least one representative from \NGO{} also attended eight of these meetings. As well as providing design suggestions, we approached these engagements through the lens of design ethnography [cite]: a qualitative research approach which, by being embedded in the design process, allowed us to gain deeper insights into the project's stakeholders' policies and practices by maintaining a critical lens on the design process and actively gathering data. These sessions were either recorded using Zoom's built-in recorder and transcribed, or a member of the research team took detailed notes during the meeting.

During the period of this study, \PC{} progressed the design through three stages. The first consisted of preliminary discussions, where we queried \PC{} and \NGO{} about the project to understand and advise on the design requirements and to gain an understanding of the company's services, business model and structure. The second was an iterative design stage, where \PCTwo{} produced several iterations of the IVR system design in Microsoft PowerPoint, detailing the `IVR flow': the menus, information and options available to the domestic workers as they interact with the system over the phone. These PowerPoint files were shared with us after each meeting. During the meetings, \PCTwo{} shared their screen to walk through the updated designs for discussion and feedback. The research team used these design decisions as probes for starting discussions around the challenges faced by the company and (less directly) their values, priorities and tensions with the project's other stakeholders. The final stage of this study saw \PC{} introduce an operational prototype, which completely supplanted the Local Guides while the actual IVR system was being built. Dubbed the `Human IVR', this acted like a `Wizard of Oz' technology prototype: in-lieu of an automated system, \PC{} used human call operators who followed the designed IVR script and the system's algorithmic recommendations to contact and interact with the domestic workers. Meetings during this stage focused on this prototype's implementation, how it was performing and what feedback \PC{} had received about it from the domestic workers.

To assess if the final platform design fulfilled \NGO{}'s project goals, the research team performed an analysis of the meeting notes, transcriptions and the system's designs to identify moments throughout which are pertinent to the research question `\textit{how did the varying agendas and values of the three stakeholders shape the experiences of the domestic workers they were seeking to empower?}' Given that the women's experiences would primarily be with the IVR system, this necessitated that the analysis focus on the women's direct and indirect interactions with it. The analysis consisted of two phases, with two different goals: i) to understand the tacit aims of the designed interaction; ii) to draw on the data to explore the underlying motivations behind it as a design decision. 
\section{Findings}

These engagements resulted in the conception and evolution of a design of an IVR system, which would be integrated into PC's existing infrastructure to support the domestic workers in interacting with the digital platform without requiring them to use new physical devices. 

\subsection{Initial Requirements Gathering}

\subsection{Design Iteration 1}

\begin{figure*}
  \centering
  \includegraphics[width=\columnwidth]{images/ivr_01_ordercall.png}
  \caption{PC2's initial PowerPoint design of the `order calls' stage, which would dial domestic workers with information about a job, and ask them for an estimated time of arrival.}~\label{fig:OrderCalls}
\end{figure*}

\begin{figure*}
  \centering
  \includegraphics[width=\columnwidth]{images/ivr_01_tracking.png}
  \caption{The initial design of the `tracking calls' stage, describing a confirmation call and three `tracking' calls to be made to domestic workers, intended to check on their progress as they travel to their client's house.}~\label{fig:TrackingCalls}
\end{figure*}

\begin{figure*}
  \centering
  \includegraphics[width=\columnwidth]{images/ivr_01_inbound.png}
  \caption{The initial design of the way the IVR system would handle inbound calls from domestic workers. The menu presented to the workers varies depending on the worker's current work state within the system.}~\label{fig:InboundCalls}
\end{figure*}

Following our first meeting, PC2 shared with us an initial design of the system in Microsoft PowerPoint. This design detailed the `IVR flow': the menus, information and options available to the domestic workers as they interact with the system over the phone. The PowerPoint consisted of three slides, each describing a different type of phone call that the system would make or receive:

\begin{itemize}
    \item The first slide (Fig \ref{fig:OrderCalls}) described the `order calls' stage: outbound calls made by the system to the workers, offering them a job and giving them some basic information about the client's location. If they accepted, the worker would then be asked to give an estimation of how long it would take for them to reach the client (an `ETA', with options of 30 minutes, one hour, or longer than one hour).
    \item The second slide (Fig \ref{fig:TrackingCalls}) detailed the `confirmation call' and three `tracking calls' which would the worker would receive from the system. The tracking calls ask the worker for updated ETA, as well as giving them the option to call PC's office or the client. These tracking calls would  be made at intervals of ten minutes after the worker received the confirmation call. 
    \item The third slide (Fig \ref{fig:InboundCalls}) described how inbound calls made by the domestic worker dialling into the system would be handled. The design offered one of three different menus to the worker depending on their current `state': if they were on their way to a client's house they would get options to hear the address details, dial the client, dial the PC office, or log the work as having started; if they were at a house on a job they would get the option of logging the work as having finished, reporting safeguarding issues, contact the emergency services or call the office; or if they didn't currently have a job assigned the could leave a rating for their last client, get information about their earnings, or contact the office.
\end{itemize}

In response to this design, we responded with a document containing a number of questions and concerns we had about the design to be addressed in the next meeting. As well as technical points of interest (e.g. `\textit{What happens if a worker's call disconnects before they give their ETA?}'), these included queries about the underlying system (e.g. `\textit{How are workers currently/previously chosen?}'; `\textit{Do the workers' ratings of the clients get used? If so, how?}'; `\textit{What's the reason for calling the domestic workers before confirming the job?}'; `\textit{How long can a worker expect to wait for the confirmation call?}'). We also highlighted three major concerns: that the tracking calls don't make allowances for distinguishing between actual travel time and when workers estimate they will arrive (`\textit{The estimated time might not be the same as travel time---the worker might have to finish house work, or organise childcare before leaving the home}'); that the frequent tracking calls are likely to be intrusive (`\textit{It’s also possible that workers will stop answering these tracking calls---being called every few minutes is likely to be annoying if they are travelling/finishing up other tasks.}'); and that the third slide's reliance on dynamic `states' may prevent the workers from being able to easily and reliably memorise the interface (`\textit{May be easier to present the same options each time (even if redundant/non-functional) for the sake of consistency, so that the user knows exactly what the options will be every time they call. This would probably also be simpler to develop and less prone to bugs.}') 



\subsection{Design Iteration 2}

\subsection{Wizard of Oz Prototype}


\section{Discussion}

\subsection{Embedded values in the system}

Analysing and querying the design of technological systems can give insights into the designer's priorities, attitudes and assumptions \citep{Star1999}. We argue that these findings highlight how the design decisions made during this process are indicative of each stakeholder's priorities and their perceptions of the women domestic workers.

There are multiple elements within the system's design and observations made during the design process which suggest that \PC{} had low expectations of the workers. For example, the `tracking calls' during the `travelling to the job' stage were perceived by \PC{} to be a necessary inclusion to prompt workers to arrive at client's houses on time---even after a high success rate during the `Human IVR' deployment (where calls the calls had been removed due to complaints from the workers), \PCTwo{} was concerned that workers would not leave on time if the calls were not coming from a human operator. Another example was the belief that the women would be confused if given the ratings of clients when offered work, and that such decisions should be made algorithmically `\textit{from the back-end}'. These low expectations and unequal provision of information contribute towards a final design where the worker has little power: as seen in analyses of other gig-economy systems \cite{martin2016, Hara2018, carlos2021, lee2015}, a lack of information flow and opaque, algorithmic control often results in a system where the worker has less power than other parties. Furthermore, there are multiple elements of the system which treat the workers as a product or commodity, such as the use of language like `Premium' and `Basic' maids in the app's interface. Such objectification, combined with the intrusive use of technology (i.e. the `tracking calls'), is reminiscent of how crowdworkers are framed: where workers are likened to `troublesome components' to be controlled, rather than human stakeholders to be designed for \cite{martin2016}.

Further inferences can be drawn regarding \PC's priorities---as noted, \PC{} are principally a for-profit company, and the system's design reflects their prioritisation of clients' interests over those of the workers. This is evident through design decisions both benign (e.g. that the worker is called to accept a job before the client is called to confirm it is legitimate; that workers receive automated calls, while customers are called by human operators) and those that have serious potential consequences (e.g. that workers' ratings of clients were not provided during IVR calls offering jobs, nor being considered by the algorithm during `Human IVR' deployment; that \PC{} had few concerns related to safety when sharing workers' phone numbers with clients). As Lee argues, supply-demand orientated algorithmic controls frequently do not account for human factors \citep{lee2015}: such issues are further evident in this system, where workers are expected to be ready to leave their home at a moment's notice, under the assumption that responsibilities such as childcare or housework obligations would have already been completed---prioritising keeping client's waiting time to a minimum. The requirement that workers stay in the client's house until \PC{} has received payment---with the onus being on the worker to follow-up on any apparent issues---again suggests that little thought had been given to the workers' experiences of the system, and serves as an example of the workers themselves being the site of engagement between the client and the company, placing additional focus and pressure upon the worker's presence and emotional performance \cite{raval2016}. These issues make it clear that the system's human-centred design focus is on the users, not the workers, further evidenced by \PC{}'s platform being built around technology that the client has access to but the worker does not, with the IVR system's existence being an attempt to work around this core inequality.

While \NGO{} made fewer contributions to the design, they were consistent in evidencing their stated goal of improving the well-being of women domestic workers. Suggestions such as the algorithm accounting for workers' ratings of clients and the inclusion of safety reminders had a clear focus on improving the workers' safety and agency. However, within the final design the focus on women's empowerment was lost: by the end of the design process, the women were no longer the priority. As the party with the platform and task of creating the implementation, \PC{} had the most control over the system's final design. As a for-profit company \PC{}'s priority was understandably to make money, and so the focus of the project naturally shifted from the women being the cause to being the product---no longer a priority, but a commodity within the design process to support an organisation's profit. We argue that these findings highlight the importance of designers being aware of their own values, priorities and assumptions: as Martin et al. argue, technology interventions introduced without care into gig-economy ecosystems have the potential to reify existing inequalities or create new ones \cite{martin2016}. Before attempting to introduce what we think are worker-centred interventions \cite{carlos2021}, designers should reflect on their own interests, and how they compare and contrast with those who will be affected by the systems we create.


\subsection{Highlighting Priorities through Technology Limitations}

As noted previously, IVR as is a technology medium with inherent design limitations that require designers to go through a process of prioritisation: menus and messages are usually prescribed and largely static, meaning that decisions be made about what information and functionality the user is given access to; and messages are read out linearly and limited in length to avoid a tedious user experience: requiring judgements be made not only on the inclusion of information, but in what order it is presented. In this regard IVR is something of a blunt instrument, where the end-user is given little ability to work outside of the confines of the design space set out by its system designer. Due to these limitations, the choices of what information and functionality is included and excluded are strong indications of the designer's priorities and assumptions. 

In this study, we have been able to provide insights into how the stakeholders' values were embedded within the final system design. A significant number of these insights were resultant of questioning and critically analysing the usage of the IVR platform, and the inclusion and omission of pieces of information being given to workers through it. Furthermore, this process was shown to be realistically applicable in a practical setting: this study was undergone within a real-world, working context, gaining insights from the ongoing development of a private for-profit company's live infrastructure. While examining this infrastructure as outside stakeholders took a considerable amount of time due to the need for familiarisation, this could be expedited if performed by investigators already aware of the design space's context. We propose that introducing constraints in system design exercises could be used by designers and engineers as an efficient way of surfacing and reflecting upon what values they are prone to embed within their systems: even in real-world, for-profit contexts. If a project does not already feature a suitably constrained medium, constraints can be contrived through abstraction: previous work has shown the benefits of using mediums such as Lego for collaboration and the communication of complex ideas \cite{Cantoni2009}. Such techniques could be utilised to critique the values embedded into system infrastructures, such as algorithms, where complexity frequently leads to practical opacity, impeding productive critical analysis. As this study has shown, this would be a particularly useful in collaborative projects where parties may have different, even conflicting, agendas. As such a relationship is seemingly almost guaranteed between for-profit gig-economy companies and their employees, we argue that such steps should be taken by any platforms looking to develop a worker-centred perspective \citep{carlos2021}.

\subsection{Breaking the cycle of disempowerment}

********* TODO What Oxfam was attempting to do was to work ‘within the gig economy system’.  Obfuscation - empowerment within a framework of disempowerment. Grey wash? There are many companies currently starting up similar programs in BD and they wanted to get in a use HT as a ‘gold star example’ of how an ethical platform could be used. But - this process revealed how working within the established paradigm wouldn’t shift the cycle of disempowerment these women constantly faced. e.g. (1) Socio-cultural perceptions embedded in the system. e.g. class - the distrust of these women, the intention to track and surveil, privacy breaches.. whilst there were continuous discussions of ‘empowerment’, the underlying values and ideals around what these women represented wasn’t addressed. In fact - these values were reflected in decisions that were embedded in the platform design. **************

Platform cooperatives: an argument for working outside the system

************* TODO Significant argument to work outside the system - disrupt. e.g. Platform cooperativism - creating platforms that reflect the values of the people. This becomes about who they are designing for. *********

accountability in design decision making - who benefits from what part

suitability of `gig' platforms and attitudes towards design for full time work with hopes of advancement
\section{Conclusion}

This paper has reported on the findings of a design ethnography undertaken with an international NGO and a for-profit gig-economy company in Bangladesh, during the design of an IVR component for use with a disadvantaged workforce of domestic workers. We argue that these design discussions regarding the IVR system acted as a `values lever': that the limited nature of the IVR format introduced a number of constraints which required the stakeholders to make design decisions which prioritised particular qualities within the system, and that discussing these decisions surfaced their values in relation to the project and their assumptions about the workers. We offer suggestions for how to introduce such values levers in technology production-focused contexts, arguing that the introduction of constraints could be a useful technique for values reflection: particularly in projects where parties may have different---even conflicting---agendas. As such a relationship is seemingly almost guaranteed between for-profit gig-economy companies and their employees, we argue that such steps should be considered during the production of any platforms looking to develop a worker-centred perspective.

%% The acknowledgments section is defined using the "acks" environment
%% (and NOT an unnumbered section). This ensures the proper
%% identification of the section in the article metadata, and the
%% consistent spelling of the heading.
\begin{acks}
Anonymised for review
\end{acks}

%%
%% The next two lines define the bibliography style to be used, and
%% the bibliography file.
\bibliographystyle{ACM-Reference-Format}
\bibliography{references}


\end{document}
\endinput

