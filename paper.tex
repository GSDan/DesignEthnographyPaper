%% The first command in your LaTeX source must be the \documentclass command.
%%%% Small single column format, used for CIE, CSUR, DTRAP, JACM, JDIQ, JEA, JERIC, JETC, PACMCGIT, TAAS, TACCESS, TACO, TALG, TALLIP (formerly TALIP), TCPS, TDSCI, TEAC, TECS, TELO, THRI, TIIS, TIOT, TISSEC, TIST, TKDD, TMIS, TOCE, TOCHI, TOCL, TOCS, TOCT, TODAES, TODS, TOIS, TOIT, TOMACS, TOMM (formerly TOMCCAP), TOMPECS, TOMS, TOPC, TOPLAS, TOPS, TOS, TOSEM, TOSN, TQC, TRETS, TSAS, TSC, TSLP, TWEB.
% \documentclass[acmsmall]{acmart}

%%%% Large single column format, used for IMWUT, JOCCH, PACMPL, POMACS, TAP, PACMHCI
% \documentclass[acmlarge,screen]{acmart}

%%%% Large double column format, used for TOG
%\documentclass[acmtog, authorversion]{acmart}

%%%% Generic manuscript mode, required for submission
%%%% and peer review
%\documentclass[manuscript,screen,review, acmconf]{acmart}
\documentclass[sigconf,anonymous]{acmart}

\usepackage{svg}
\usepackage{csquotes}

\newcommand{\NGO}{\textit{NGO}}
\newcommand{\NGOOne}{NGO1}
\newcommand{\PC}{\textit{GigCo}}
\newcommand{\PCOne}{GigCo1}
\newcommand{\PCTwo}{GigCo2}

%% Rights management information.  This information is sent to you
%% when you complete the rights form.  These commands have SAMPLE
%% values in them; it is your responsibility as an author to replace
%% the commands and values with those provided to you when you
%% complete the rights form.
\setcopyright{acmcopyright}
\copyrightyear{2018}
\acmYear{2018}
\acmDOI{10.1145/1122445.1122456}

%% These commands are for a PROCEEDINGS abstract or paper.
\acmConference[CHI'22]{CHI 2022}{April 30--May 06, 2022}{New Orleans, LA}
\acmBooktitle{ACM CHI 2022 Conference on Human Factors in Computing Systems,
  April 30--May 06, 2022, New Orleans, LA}
\acmPrice{15.00}
\acmISBN{978-1-4503-XXXX-X/18/06}


%%
%% Submission ID.
%% Use this when submitting an article to a sponsored event. You'll
%% receive a unique submission ID from the organizers
%% of the event, and this ID should be used as the parameter to this command.
%%\acmSubmissionID{123-A56-BU3}

%%
%% The majority of ACM publications use numbered citations and
%% references.  The command \citestyle{authoryear} switches to the
%% "author year" style.
%%
%% If you are preparing content for an event
%% sponsored by ACM SIGGRAPH, you must use the "author year" style of
%% citations and references.
%% Uncommenting
%% the next command will enable that style.
%%\citestyle{acmauthoryear}

%%
%% end of the preamble, start of the body of the document source.
\begin{document}

%%
%% The "title" command has an optional parameter,
%% allowing the author to define a "short title" to be used in page headers.
\title{Critically Engaging with Embedded Values through Constrained Technology Design}

%%
%% The "author" command and its associated commands are used to define
%% the authors and their affiliations.
%% Of note is the shared affiliation of the first two authors, and the
%% "authornote" and "authornotemark" commands
%% used to denote shared contribution to the research.
\author{Anonymous for review}
\affiliation{%
  \institution{Anon}
  \streetaddress{Anon}
  \city{Anon}
  \country{Anon}}
\email{Anon@Anon.org}

\author{Anonymous for review}
\affiliation{%
  \institution{Anon}
  \streetaddress{Anon}
  \city{Anon}
  \country{Anon}}
\email{Anon@Anon.org}

\author{Anonymous for review}
\affiliation{%
  \institution{Anon}
  \streetaddress{Anon}
  \city{Anon}
  \country{Anon}}
\email{Anon@Anon.org}

\author{Anonymous for review}
\affiliation{%
  \institution{Anon}
  \streetaddress{Anon}
  \city{Anon}
  \country{Anon}}
\email{Anon@Anon.org}

%%
%% By default, the full list of authors will be used in the page
%% headers. Often, this list is too long, and will overlap
%% other information printed in the page headers. This command allows
%% the author to define a more concise list
%% of authors' names for this purpose.
\renewcommand{\shortauthors}{Anon et al.}

%%
%% The abstract is a short summary of the work to be presented in the
%% article.
\begin{abstract}
The rise of gig-economy platforms has highlighted the impact platform and algorithm design can have upon workers’ experiences. This paper reports on an extended series of collaborative design engagements with a private company and an international NGO during the production of an IVR component for a gig-economy platform. We present the findings of a design ethnography undertaken during this process and discuss how design decisions reflect how each party’s values, motivations and assumptions are embedded within the final technology. We argue that there exists a need for simple methods to assist designers and engineers to surface and critically engage with the potentially disparate values, priorities and assumptions held by themselves and system stakeholders, and contribute suggestions for how this can be done during the production of real-world systems through the application of constrained design as a values lever.
\end{abstract}

%%
%% The code below is generated by the tool at http://dl.acm.org/ccs.cfm.
%% Please copy and paste the code instead of the example below.
%%
\begin{CCSXML}
<ccs2012>
 <concept>
  <concept_id>10010520.10010553.10010562</concept_id>
  <concept_desc>Computer systems organization~Embedded systems</concept_desc>
  <concept_significance>500</concept_significance>
 </concept>
 <concept>
  <concept_id>10010520.10010575.10010755</concept_id>
  <concept_desc>Computer systems organization~Redundancy</concept_desc>
  <concept_significance>300</concept_significance>
 </concept>
 <concept>
  <concept_id>10010520.10010553.10010554</concept_id>
  <concept_desc>Computer systems organization~Robotics</concept_desc>
  <concept_significance>100</concept_significance>
 </concept>
 <concept>
  <concept_id>10003033.10003083.10003095</concept_id>
  <concept_desc>Networks~Network reliability</concept_desc>
  <concept_significance>100</concept_significance>
 </concept>
</ccs2012>
\end{CCSXML}

\ccsdesc[500]{Computer systems organization~Embedded systems}
\ccsdesc[300]{Computer systems organization~Redundancy}
\ccsdesc{Computer systems organization~Robotics}
\ccsdesc[100]{Networks~Network reliability}

%%
%% Keywords. The author(s) should pick words that accurately describe
%% the work being presented. Separate the keywords with commas.
\keywords{design ethnography, digital civics}


%%
%% This command processes the author and affiliation and title
%% information and builds the first part of the formatted document.
\maketitle

\section{Introduction}

The use of online platforms to find small jobs has seen a worldwide explosion in popularity in recent years \cite{Taylor2017, islam2019, Wood2019}. Dubbed the `gig-economy', it exists as a labour-based subset of the `sharing economy', where workers capitalise on their skills and spare time as underused assets for income generation, often in addition to other commitments \cite{Balaram2017}. The pace at which the gig-economy has grown and `disrupted' markets has presented its own issues, however, and employment law has struggled to keep up \cite{Minter2017}, resulting in fears of worker exploitation due to a lack of necessary protections and safety nets \cite{Balaram2017}. However, any laws which do get introduced also need to protect the flexibility and autonomy offered to workers by these platforms, which made them appealing to so many in the first place \cite{Wood2019}. Similarly, the technologies which power the gig-economy need to account and balance the disparate needs of two primary stakeholder groups: their customers (and, in turn, company profit), and their workers (whose requirements may vary, based on context and field of work \cite{carlos2021}). An understanding of how designers' assumptions and priorities are embedded into these systems is therefore essential, as such design decisions have the potential to both benefit and hinder these stakeholders.

While waiting for this official legislation and regulation of the gig-economy to be introduced, there have been calls for non-state actors---such as unions and NGOs---to negotiate to formal agreements with gig-economy companies, providing short-term benefits to workers and highlighting issues with existing legislation \cite{Minter2017}. This paper engages within this context, and presents findings from an extended series of engagements from collaborative design engagements with an international advocacy group (\NGO{}) and a start-up company (\PC{}) which runs a gig-economy platform for disadvantaged domestic workers in Dhaka, Bangladesh. The product of these engagements is an Interactive Voice Response (IVR) system, designed to connect the predominantly offline domestic workers to \PC{}'s digital infrastructure in order to distribute and manage jobs. We present the results of a design ethnography undertaken during this process, discussing the system's design and how both party's values, motivations and assumptions have been embedded within it. We contribute our insights from this series of design engagements, suggestions for how designers and engineers can critically engage with their own values as a part of the design process, and discuss the potential pitfalls non-state actors face when attempting to formally advocate for workers while working within a framework of disempowerment.
\section{Related Work}
\section{Context}

This project came about through a three-way partnership between: the Bangladeshi branch of an international NGO (anonymised as `\NGO{}'); the research team, members of a HCI research group based in Australia; and \PC{} (anonymised), a gig-economy startup company based in Dhaka, Bangladesh (BD).

\NGO{} was in the midst of a project aiming to improve the well-being of women domestic workers in BD, who are typically hired to cook, clean, do laundry or even care for the children of a household. Estimates range between there being 2 million \cite{DWRN2011} and 4 million \cite{Ashraf2019} domestic workers in BD, with around 80\% being women or girls \cite{Ashraf2019}. Despite their prevalence, domestic work in BD was only recognised as an informal profession in 2015, with the introduction of the Domestic Workers Protection and Welfare Policy (DWPWP): prior to this, domestic workers were not entitled to time off, were not legally assured `fair' wages (85\% live under the poverty line \cite{BILS2015}), and there were few legal protections from abuse and harassment within their places of work \cite{IDWF2015}. However, adherence to these new policies has been inconsistent \cite{islam2016}, with a perceived lack of policy enforcement and reports of abuse still frequent \cite{DailyStar2018}. Within this context, \NGO{} aims to provide women domestic workers with skills training for formal job opportunities, to increase their awareness of their rights, and to support the BD government's capacity to enforce and monitor the implementation of the DWPWP. 

As discussed, gig-economy platforms are increasingly common in BD \cite{Ahmed2020}. In-fitting with a prevalent trend of techno-solutionist social interventions within BD (e.g. \cite{hasnayen2016, Faroqi2019}), \NGO{} chose this as a sector through which to promote their agenda and partnered with the gig-economy startup \PC{}. Aiming to be `the Uber of domestic workers', \PC{} runs an app-based service through which customers can request a domestic worker using their smartphone. In exchange for configuring their business towards the promotion of the DWPWP and the empowerment of domestic workers, \NGO{} are assisting \PC{} with the recruitment and training of several thousand domestic workers to support higher rates of pay. \PC{}'s `About Us' page notes that they `\textit{dream to build an ecosystem where every family employs trained, skilled and verified domestic helpers}' and that they aim to provide a `\textit{secure workplace for millions of domestic helpers through our platform and establish "domestic work" as a dignified profession}'. As such, \PC{}'s priority was to create a platform which would enable them to: i) meet commercial interests, and ii) empower domestic workers.

Many women in Bangladesh live in a digital divide driven by gender and socio-economic status \cite{Genilo2015}, and the vast majority of these workers lack access to smartphones: usually only having a `feature phone' which can place calls and send SMS (although many are illiterate, making SMS of limited value). As a result, the majority of \PC{}'s workforce are unable to directly interact with the digital infrastructure to respond to client requests. In response, \PC{} initially introduced `local guides': another tier of gig-economy workers who could read and had access to a smartphone, taking orders through the app and forwarding their details to domestic workers through phone calls. However, \PC{} reflected that this process was inefficient and problematic, easily manipulated by a local guide's preferences or even corruption:

\begin{displayquote}
"There was always an option to manipulate by thinking ‘Oh, I like this domestic worker, I want to give her the job.’ We provide the local guide with a list of ten domestic workers in order of [\PC{}'s] preference, but the local guide could select the tenth worker because it was a manual process."
\end{displayquote}

In late 2020, \PC{} and \NGO{} decided to explore alternative, automated solutions to bridge the domestic workers with the digital infrastructure. They decided to create an Interactive Voice Response (IVR) system: an automated telephone-based system which can be interacted with by callers through button presses. An automated IVR system was deemed to be more `objective', as excluding the local guides from the process removed a layer of human judgement. As a research team with experience of IVR technologies and an existing relationship with \NGO{}, we offered consultation and advice during the design process of this new system in exchange for access to the project as a research context.
\section{Methodology}

This study took place over the course of 11 months, performed remotely due to the global pandemic. During this time, the research team hosted 12 Zoom meetings with members of \PC{} (\PCOne{} \& \PCTwo{}), with each meeting running for an average of one hour and all parties providing design suggestions. At least one representative from \NGO{} also attended eight of these meetings. These sessions were either recorded using Zoom's built-in recorder and transcribed, or a member of the research team took detailed notes during the meeting. As argued by Winner, the designs of systems and infrastructures can have implicit or explicit political qualities embedded within them \cite{winner1980}: Star suggests that the ethnographic study of such infrastructures can provide insights into the designers' decision making priorities \cite{Star1999}. By its nature, IVR is a format well suited to this type of investigation: limitations such as restricted message length and linear, sequential menus require that system designers prioritise access to some information and functionalities over others. With this in mind, we approached this study through the lens of design ethnography: a qualitative research approach which, by being embedded in the design process, allowed us to gain deeper insights into the project's stakeholders' policies and practices by maintaining a critical lens on the design process through active data collection. 

During the period of this study, \PC{} progressed the design through three stages. The first consisted of preliminary discussions, where we queried \PC{} and \NGO{} about the project to understand and advise on the design requirements and to gain an understanding of the company's services, business model and structure. The second was an iterative design stage, where \PCTwo{} produced several iterations of the IVR system design in Microsoft PowerPoint, detailing the `IVR flow': the menus, information and options available to the domestic workers as they interact with the system over the phone. These PowerPoint files were shared with us after each meeting. During the meetings, \PCTwo{} shared their screen to walk through the updated designs for discussion and feedback. The research team used these design decisions as probes for starting discussions around the challenges faced by the company and (less directly) their values, priorities and tensions with the project's other stakeholders. The final stage of this study saw \PC{} introduce an operational prototype, which completely supplanted the Local Guides while the actual IVR system was being built. Dubbed the `Human IVR', this acted like a `Wizard of Oz' technology prototype: in-lieu of an automated system, \PC{} used human call operators who followed the designed IVR script and the system's algorithmic recommendations to contact and interact with the domestic workers. Meetings during this stage focused on this prototype's implementation, how it was performing and what feedback \PC{} had received about it from the domestic workers.

To assess if the final platform design fulfilled \NGO{}'s project goals, the research team performed an analysis of the meeting notes, transcriptions and the system's designs to identify moments throughout which are pertinent to the research question `\textit{how did the varying agendas and values of the three stakeholders shape the experiences of the domestic workers they were seeking to empower?}' Given that the women's experiences would primarily be with the IVR system, this necessitated that the analysis focus on the women's direct and indirect interactions with it. The analysis consisted of two phases, with two different goals: i) to understand the tacit aims of the designed interaction; ii) to draw on the data to explore the underlying motivations behind it as a design decision. 
\section{Findings}

\begin{figure*}
  \centering
  \includesvg[width=1.5\columnwidth]{images/ht_system.svg}
  \caption{The designed workflow for a \PC{} order. The client requests a worker through the smartphone app (1). \PC{}'s algorithm chooses a suitable worker (2) and rings them to see if they will take the order (3). If so, the client is called by customer support, to ensure the order is genuine (4). If it is, the worker receives a confirmation call from the system (5) and is asked to give an ETA and leave towards the client's house. During the journey, the worker is called up three times to get an updated ETA (6). Upon arrival, the worker calls the system to log the work as having started (7) and completes the job (8). After finishing, the worker calls into the system to log the work as finished (9). At any time, the worker can call the system to get information about a current job or their wages, get help from the \PC{} office or emergency services, or to contact the client (X). }~\label{fig:WorkFlow}
\end{figure*}

The final design for the system's workflow, as created by \PCTwo{} with feedback from \NGOOne{} and the research team, is described in Figure \ref{fig:WorkFlow}. This section details each of the core stages of this process, and the reasoning for them as given by the participating stakeholders from \PC{} and \NGO{} during our meetings.

\subsection{Choosing a Worker}

After a job request was received from a client, \PC{} needed to assign a domestic worker to fulfill it. In the designed IVR system, the process of deciding the most suitable worker was handled by an algorithm (Figure \ref{fig:WorkFlow}.2), and the chosen worker would receive an automated call offering them the job (Figure \ref{fig:WorkFlow}.3). After agreeing to take the job, the worker is asked to choose their estimated time of arrival (ETA) from multiple options, and told to wait for a confirmation phone call before leaving towards the client's house. Prior to this automated design, this was handled by the local guides, who would be given details of the job and asked to call and assign a suitable worker to it. However, the local guides' choices were not always in-line with PC's priorities:

\begin{displayquote}
\textbf{\PCTwo{}}: "When it was local guides, our service often saw failure when trying to get a domestic worker. It was not always within one or two kilometres: sometimes they always wanted workers to get to work, whether it is 1, 2 or even 3 kilometres. Or sometimes they used the logic that `\textit{I have to distribute the work equally---I have sent a domestic worker to a job in the morning, now, even though this new job is near that worker’s house, I will send another worker to distribute it equally.}’ But now, in our system you cannot do that."
\end{displayquote}

The \PC{} and \NGO{} stakeholders were also aware that this system was open to potential manipulation and exploitation by the local guides:

\begin{displayquote}
\textbf{\PCOne{}}: "When we had local guides in place, they assigned domestic workers manually, so there was always an option to manipulate: ‘\textit{Oh, I like this domestic worker, I want to give her the job.}’ [...] We provide the local guide with a list of ten domestic workers ordered based on the requirements, like from one to ten---the local guide can still assign the tenth domestic worker because it’s a manual process."
\end{displayquote}

Instead of relying on human decision making, the new system uses an algorithm which prioritised the workers based on their location, user rating, and system rating:

\begin{displayquote}
\textbf{\PCTwo{}}: "Firstly our app detects the domestic workers within a 1km radius of the customer’s home, and [then it takes into account] the rating by our users. [...] The next rating is the `system rating'. This is updated based on the responses from the domestic worker to offers of work: if she had been rejecting orders each and every time it will develop a bad rating for her. For new domestic workers---you raised this concern---they start with an initial rating of 100. If she performs well, it will be around that 100, 98 or 97… if she performs bad it will be dropping. This way, new domestic workers will always receive preference to be offered work."
\end{displayquote}

After a suggestion from \NGOOne{}, this was adapted to also take into account historical ratings of the client from workers: 

\begin{displayquote}
\textbf{\NGOOne{}}: "As [\PCTwo{}] mentioned, one rating will be from the employer’s side---if she gets five [stars] she’ll be called again. We are also taking ratings from the employee level: how she feels working there. Are you going to consider this? Because it might be that she might not be comfortable with working there---it might be that I’m a bad person, and give her five stars to get her again, but she’s not comfortable with me. So we need to..."

\textbf{\PCTwo{}}: "Both sides of the coin. Yes. We should make it the first point: if the domestic worker does not want to go there, we should not consider her..."
\end{displayquote}

During the `human IVR' deployment, \PCTwo{} described the prototype's decision-making process as being `half human brain, half computer':

\begin{displayquote}
\textbf{\PCTwo{}}: "It was like a full human brain before---our [local guides] were using their brain to find a domestic worker, with slightly different logic. But now it is one human brain and one computer: the computer is receiving the order, is finding the domestic worker. The other human brain is just interacting after that---‘\textit{Ok, you are the chosen one}’---they’re getting a list of five or ten. They do not even know these domestic workers. I think it will just be the one computer when it is IVR, without the human brain. We are almost half way, or more than half way."
\end{displayquote}

However, the domestic workers reacted negatively to this new process, finding it limiting and uncomfortable to talk to a scripted stranger:

\begin{displayquote}
\textbf{\PCTwo{}}: "We have had some feedback. It was easier for them to interact with the `complete human brain’: they could interact in an unlimited way, but now it’s a limited logic, and they want to see the face of the customer support---they have seen the face of their local guide. So they used to feel at ease, but now they feel like they get a call from a random place, and they forget the order. Sometimes they even complain: ‘\textit{we can’t understand their language.}' It's not a problem of accent or language, it’s about thinking ‘\textit{I do not know him, so I do not get his message.}’"
\end{displayquote}

When the system calls the prioritised worker, it gives a basic overview of the job: the client's name, the area they live in, how long the job will take, and how much they will get paid. The research team suggested that the worker could also benefit from being informed of the client's rating by other domestic workers. \PC{} were initially hesitant, noting that the rating might confuse the workers, and opining that decisions about which clients should be served should be made by \PC{} in the back-end: 

\begin{displayquote}
\textbf{\PCTwo{}}: "Right, but I’m worried that the domestic worker might not understand the rating system to be able to assess the client. So I think we need to do the screening from the back end."
\end{displayquote}

Despite appearing to agree and them earlier noting that workers' wants should be considered, the feature didn't make it into the final design:

\begin{displayquote}
\textbf{R2}: "It doesn't have to be a number: it could be a word, like ‘\textit{this client has been rated excellent}’."

\textbf{\PCTwo{}}: "Right. We could use a word, we could do that."
\end{displayquote}


\subsubsection{Confirming the Job}

Once a worker has been found, one of PC's customer support agents then calls the client to confirm that the job is legitimate (Figure \ref{fig:WorkFlow}.4): \PCTwo{} noted that clients frequently create jobs within the smartphone app to test that the system works, and so these test jobs need to be identified before telling the assigned worker to leave for their house. They specified that this should even happen for repeat clients, not just ones new to the application. When we asked why this confirmation is made after first finding a worker, they noted that they didn't want to confirm with a legitimate customer only to then cancel the job if they couldn't find a suitable worker: 

\begin{displayquote}
\textbf{\PCTwo{}}: "We try to find a domestic worker for the required job first, and then we go to the user to confirm the order, because [...] we do not want to call the user without first confirming that we have a domestic worker in place to serve him. It will make him dissatisfied---we have called the user but then we won’t be able to provide a domestic worker. So, we want to make sure that we have a domestic worker in place so that if he or she wants to take the order, we can provide it."
\end{displayquote}

After confirming the job is legitimate, the worker will then receive a second automated `confirmation call' from the system (Figure \ref{fig:WorkFlow}.5). As a condition of their partnership, \NGO{} required that this confirmation call include a message outlining precautions which would assist in the worker's safety:

\begin{displayquote}
\textbf{\NGOOne{}}: "Know that [\PC{}] is practicing a very good thing with a few conditions: when the domestic worker gets the confirmation, she will be informed [to only work] if  there is a woman in the workplace, if she feels safe, and to always wear a mask. So three instructions will be there, so that we can take three actions to make her safer."
\end{displayquote}

For the `Human IVR' prototype, these warnings had been formalised into `checkpoints'---standards which the worker had been instructed to be sure had been met, otherwise they were instructed to call \PC{} to cancel the order and return home:

\begin{displayquote}
\textbf{\PCTwo{}}: "We have several ‘checkpoints’ when serving the order: whether it is a family house, whether a female member is present in the house while the worker will be working."
\end{displayquote}

\subsubsection{Tracking the Worker}

While travelling to the client's house, the worker receives up to three automated `tracking calls' (Figure \ref{fig:WorkFlow}.6): the first is made 15 minutes after they received the confirmation call, and asks if the worker has left yet; another call is made after two thirds of the given ETA has expired, and asks for an updated ETA; and the final call is made 10 minutes after the ETA if the worker has not yet logged starting work. As the workers are likely to be departing from their own house, we queried if it's possible that they are likely to be in the middle of other tasks when orders come through, such as housework, cooking or childcare:

\begin{displayquote}
\textbf{\PCTwo{}}: "In our training, we’ll be telling them to ‘\textit{please be ready when you get an order call, and set out as soon as you get the confirmation call}’. So it’s our expectation that she’ll be ready and waiting for the confirmation call to get out. [...] We are training them to be done with all of their work from their end in the morning, and wait for the job all day long, we will be giving you one after one."
\end{displayquote}

We raised a concern that these calls are likely to feel intrusive, and asked for clarification on the intention behind them. \PCTwo{} claimed they were necessary in order to `push' the workers to get to the client on time: 

\begin{displayquote}
\textbf{\PCTwo{}}: If they have received an order we have to keep pushing them if she is already on the way, because we have to maintain a certain time---on the user’s side we have already committed them at the start of the order that it will take 30 minutes, 60 minutes, based on the estimation of the domestic worker. So we don’t need to make this [time] any longer than this, so we keep pushing them if they are already on the way."
\end{displayquote}

\PCOne{} seemed to believe that the number of calls was necessary, and might even need to be increased for the final design:

\begin{displayquote}
\textbf{\PCOne{}}: "When we test the system we might find that we need more tracking calls, maybe we don’t need this many, even."
\end{displayquote}

After implementing the `Human IVR' model, \PC{} received negative feedback from their workers about the tracking calls:

\begin{displayquote}
\textbf{\PCOne{}}: "Maybe we need to call less in the IVR, because a lot of calls [hamper] the workers: maybe they’re on their way, on the bus, they really don’t want to receive a lot of calls. [...] We don’t need to call that much. I mean, without calling the domestic workers, after half of the time, almost all of them went to the customer’s house. So I think we have three tracking calls on the [IVR design] flow, I think it can be less. Maybe one."
\end{displayquote}

Despite the workers getting to clients on time without all three tracking calls, \PCTwo{} was sceptical that this success would carry over to the full IVR system:

\begin{displayquote}
\textbf{\PCTwo{}}: "We know you were not on the side of the tracking calls, we have three tracking calls in the IVR plan, but currently we only have one tracking call in the current system. [...] Because we still have the human brain interacting here, ensuring that the domestic worker is [on the call], talking. So, it’s almost 100\% that they are going out to the client’s house in proper time. But I’m worried that in the case of the IVR, it will be a little lower, or much lower. It may be that we need at least one tracking call at the start [of the journey]---whether they have gotten out or not."
\end{displayquote}

\subsubsection{Calls from the Worker}

Once at the client's house, workers log themselves as having started (Figure \ref{fig:WorkFlow}.7) and finished (Figure \ref{fig:WorkFlow}.8) the work by calling into the service. Previously, this had been the responsibility of the client, who would do it through the app. \NGOOne{} noted that clients had been known to abuse this power to avoid paying workers for their actual time worked, and that the new system is designed in response:

\begin{displayquote}
\textbf{\NGOOne{}}: "Sometimes the customers are very tricky---he will not say anything [about the time being over] because she is busy doing something. So after finishing the job, the customer will get an SMS saying that she has worked her minutes, and any longer she would get paid extra taka. And after another 5 minutes, the worker will get a call saying `\textit{you have completed your work hours, you will get this amount}.’"
\end{displayquote}

At any time, including while at a client's house, workers were able to get contact either the \PC{} office or emergency services for help (Figure \ref{fig:WorkFlow}.X). However, it appeared that the aforementioned pre-work `checkpoints' meant that after starting work, very few calls came from workers complaining about clients:

\begin{displayquote}
\textbf{\PCOne{}}: "We don’t serve bachelor houses, and family houses need to have a female member, so our domestic workers are instructed to check if there is a female member or not. What happens if they see the house doesn't have a female member? They come back. We have a status called `maid return’. It happens, maybe every day two or three times: where they don’t go in the customer's house for not having a female member. But after [starting] serving to the customers---we really don’t get any calls about safety issues. It’s almost 0\%."
\end{displayquote}

After the job has been marked as completed, the worker is given the option of leaving feedback about the client. \NGOOne{} was keen that the workers be given multiple ways of giving feedback, as well as assurances of confidentiality:

\begin{displayquote}
\textbf{\NGOOne{}}: "Considering the feedback mechanism: [...] she can leave a rating, she can leave a voice call, but there should be a voice which says ‘\textit{what you are sharing here will be fully confidential and will only be used for official purposes}.’ So that she can raise her voice herself."
\end{displayquote}

However, for the `Human IVR' phase, workers' feedback submissions were not being fully taken into account by the algorithm:

\begin{displayquote}
\textbf{R1}: "One of the last times we talked, we talked about how workers could rate, or even blacklist, clients. Is that in place?"

\textbf{\PCTwo{}}: "They can actually---they can blacklist the client by calling us."

\textbf{R1}: "And does the algorithm take that into account---does the dashboard interface take into account that the worker has blacklisted that client?"

\textbf{\PCTwo{}}: "No, not yet. But so that the client is getting the domestic worker that they want, we have a simple technique here---we are encouraging our clients to give 5 stars to his or her preferable domestic worker, and when a worker gets a five star from the client, she will be in first place [of the recommendation algorithm]. But it is not in action that the domestic worker will not get the same [client] if she doesn't want to, through a phone call, so we need to work on that."
\end{displayquote}

As well as get help from the company office or emergency services, the IVR design gave workers the ability to get in contact with the client (to allow workers to say they've arrived outside the property, or to ask for more specific directions). The IVR system acts as an intermediary, so that neither party can see the other's phone number---the original system instead gave clients the workers' phone number inside of the app. When we raised the potential safety and privacy issues regarding giving clients access to workers' phone numbers, PC1 and \PCOne{} claimed that it hadn't been an issue and weren't concerned. However, they were interested in implementing this intermediary system as it could potentially lower the worker `bounce rate': when clients use the workers' details to hire them directly, outside of the \PC{} platform.


TODO

\begin{displayquote}
\textbf{\PCOne{}}: Yes, we always have a shortage of domestic workers. [laughs] That’s why we started working with Oxfam - they promise us the system to train domestic workers, but beside this, we always have a shortage of domestic workers. In this current COVID time, they’ve started working with us - they’re more willing to join HelloTask. But other times, they don’t work with us - the whole system is a hassle for them. They’re very good [and professional?], so why would they work with HelloTask?
\end{displayquote}
\section{Discussion}

These findings highlighted not only that elements of both \NGO{} and \PC{}'s values and assumptions were embedded into the final design of the system, but that process of designing the IVR provided an excellent entrypoint into discussions around these values. This section discusses what values and assumptions were surfaced, how the IVR acted as a values lever to enable these discussions, and provides recommendations for how constraints can be utilised as design tools for engaging with stakeholder values.

\subsection{Embedded Values in the System}

Despite \PC{} and \NGO{} believing that the IVR system would offer more objectivity than the `local guides', infrastructures can contain their own biases which are often informed by the values and priorities of those involved in the design process \citep{winner1980}. Likewise, analysing and querying the design of technological systems can give insights into the designer's priorities, attitudes and assumptions \citep{Star1999}. We argue that these findings highlight how the design decisions made during this process are indicative of each party's priorities and their perceptions of the domestic workers.

Firstly, there are multiple elements within the system's design and observations made during the design process which suggest that \PC{} had low expectations of the workers. For example, the `tracking calls' were perceived by \PC{} to be a necessary inclusion to prompt workers to arrive at client's houses on time---even after a high success rate during the `Human IVR' deployment (where the calls had been removed due to complaints from the workers), \PC{} was concerned that workers would not leave on time if the calls were not coming from a human operator. As in crowdsourcing platforms, this intrusive use of technology for monitoring frames workers as `troublesome components' to be controlled, rather than human stakeholders to be designed for \cite{martin2016, Irani2013}. \PC{} also believed that the women would be confused if given the ratings of clients when offered work, and that such decisions should be made algorithmically `\textit{from the back-end}'. Furthermore, while clients had access to workers' phone numbers, workers were only able to directly connect to clients once by routing through the IVR system. These low expectations and unequal provision of information and control contribute towards a final design where the worker has little power: as seen in analyses of other gig-economy systems \cite{martin2016, Hara2018, carlos2021, lee2015}, a lack of information flow and opaque, algorithmic control often results in a system where the worker has less power than other parties. 

Further inferences can be drawn regarding \PC's priorities---as noted, \PC{} are principally a for-profit company, and the system's design reflects their prioritisation of clients' interests over those of the workers. This is evident through design decisions both benign (e.g. the worker is called to accept a job before the client is called to confirm it is legitimate; that workers receive automated calls, while customers are called by human operators) and those that have serious potential consequences (e.g. that workers' ratings of clients were not provided during IVR calls offering jobs, nor being considered by the algorithm during `Human IVR' deployment; that \PC{} had few concerns related to safety when sharing workers' phone numbers with clients). As Lee argues, supply-demand orientated algorithmic controls frequently do not account for human factors \citep{lee2015}: such issues are further evident in this system, where workers are pushed through repeated phone calls during travel, prioritising keeping clients' waiting time to a minimum. The requirement that workers stay in the client's house until \PC{} has received payment again suggests that little thought had been given to the workers' experiences of the system, and serves as an example of the workers themselves being the site of engagement between the client and a corporate entity: placing additional focus and pressure upon the worker's physical presence and emotional performance \cite{raval2016}. Finally, that \PC{}'s platform is built around technology that the client has access to but the worker does not---with the IVR component's existence being an attempt to navigate this core inequality---highlights that the system's human-centred design focus is on the users, not the workers.

While \NGO{} made fewer contributions to the design, they were consistent in evidencing their stated goal of improving the well-being of women domestic workers. Suggestions such as the algorithm accounting for workers' ratings of clients and the inclusion of safety reminders had a clear focus on improving the workers' safety and agency. However, within the final design the focus on women's empowerment was lost, and the women were no longer the priority. As the party with the platform and task of creating the implementation, \PC{} had the most control over the system's final design. Despite producing some concrete improvements towards the upholding of the standards set out in the DWPWP, as a for-profit company \PC{}'s priority was understandably to make money, and so the focus of the project naturally shifted from the women being the cause to being the product: no longer the priority, but a commodity in a capitalist process. We argue that these findings reinforce the importance of designers being aware of each other's values, priorities and assumptions: as Martin et al. argue, technology interventions introduced without care into gig-economy ecosystems have the potential to reify existing inequalities or create new ones \cite{martin2016}. We also posit that the lack of consultation with the workers also meant that \PC{} and \NGO{}'s existing assumptions and values were more likely to be embedded in the design. Before attempting to introduce what we think are worker-centred interventions \cite{carlos2021}, we as designers and engineers need to reflect on our own interests, and how they compare and contrast with those who will be affected by the systems we create. Doing so requires an understanding of the values and interests of all involved.


\subsection{Technology Constraints as Values Levers}

As noted previously, IVR is a technology medium with inherent design limitations that require designers to go through a process of prioritisation. Menus and messages are usually prescribed, linear, largely static and should be limited in length to support cognition \cite{Suhm2008}, meaning that decisions be made about what information and functionality the user is given access to. Such limitations require judgements be made not only on the inclusion of information, but in what order it is presented. Furthermore, most traditional IVR systems act as blunt instruments, where the end-user is given little ability to work outside of the confines of the design space set out by its system designer. Due to these limitations, the choices of what information and functionality is excluded and how included elements are ordered are strong indications of the designers' priorities and assumptions about the end user. 

As such, the process of designing the IVR component itself became a values lever \cite{Shilton2013}. We argue that Shilton's given examples of limitations placed upon designers and engineers by platforms and institutions \cite{shilton2018, shilton2019} can be directly compared to the process of designing an IVR system: all of these sets of limitations acted as entry points for reflection and value discussions. In this study, we have been able to provide insights into how the stakeholders' values were embedded within the final system design, as direct results of questioning and critically analysing presented designs of the IVR platform: querying the inclusion, prioritisation, and omission of pieces of information and functionality accessible to workers through it, and how such decisions affect and reflect other aspects of the platform, such as the worker selection algorithm. As these discussions took place during the design process, the design itself came to be shaped by them and the values of those involved: examples include the inclusion of \NGO{}'s safety measures; \PC{} taking our feedback relating to prioritising quick access to emergency support over more commonly used menu options related to active jobs; and the algorithm's worker selection metrics, with \NGO{} and \PC{} negotiating a new solution that prioritised and including limits on worker distance to a mutually beneficial result.

However, this study also showed that the introduction of values levers is not a `magic bullet' to produce a value and worker-centred design: despite extensive dialogue between the parties, our findings showed the final design to still more closely reflect the values of \PC{} over \NGO{} or the research team. Examples of issues discussed within design sessions but as-yet unimplemented include: our concerns relating to the application giving clients the workers' phone numbers, and \NGO{}'s suggestion that a client's rating by workers should be accounted for by the worker selection algorithm. While these topics were raised through the IVR design process as points for critical discussion, we argue that the fact that these issues and suggested improvements were not acted upon is a separate issue: resulting from \PC{} having significantly greater control over the design and implementation than \NGO{} or ourselves. 

\subsection{Recommendations for Introducing Values Levers in Design Tools}

While the use of artefacts as design tools (e.g. \cite{GrowAGame, Alshehri2020}) is widely valued by UX designers and researchers as a method of prompting critical and creative discussions between stakeholders, such reflective processes are often undesirable within pragmatic, production-focused development environments \cite{Shilton2013}. We argue that not only are constraints-based values levers useful tools for reflecting on the values and assumptions being embedded into systems, but that they can also be realistically applied in production-focused contexts: this study was undertaken in partnership with a private for-profit company, gaining insights from their ongoing development of live infrastructure. We propose that introducing constraints in system design exercises could be used by system designers and engineers as an efficient way of surfacing and reflecting upon what values they are prone to embed within their platforms, even in real-world, for-profit contexts. Based on our insights gained from this study, we offer suggestions for how system design exercises could effectively configured to introduce values levers:

\subsubsection{Include Representatives from All Stakeholders}

An obvious flaw of this project was the lack of consultation or co-design with the system's main stakeholders: the domestic workers themselves, whose empowerment was meant to be the project's primary goal. Beyond advocacy by \NGO{}, the workers only became included in the design process during the late `Human IVR' stage, where their negative feedback relating to the tracking calls surprised \PC{}. Including more stakeholders' domain knowledge, values, and motives within the critical reflection process may be challenging due to increased odds of encountering tension points; however, as shown, these would likely have been encountered anyway by the introduction of the final design. Highlighting, navigating and negotiating through all stakeholders' values and assumptions earlier in the design process to find commonalities and confluences would likely have produced a more suitable final product. Furthermore, explaining design decisions across disciplinary barriers can serve as a form of values lever in itself \cite{shilton2018}, and including workers in the design of the platforms they work in could help reduce the technology's perceived opacity: preventing mistrust and reducing the impact of power asymmetry \cite{lee2015, martin2016, carlos2021}.


\subsubsection{Introduce Constraints to Require Explicit Decision-Making}
In this paper, we have argued that introducing design materials which feature resource-based constraints can serve as an effective way of surfacing stakeholders' values by forcing them to make decisions. The empowerment of domestic workers was shown to be one of multiple values which had to be negotiated alongside other (i.e. commercial) interests: a fact highlighted by the limitations of the IVR medium, which required stakeholders to prioritise particular information and functionality above others. While IVR was already conveniently constrained by its nature, new constraints can be contrived through abstraction if necessary: previous work has shown the benefits of using mediums such as Lego as design tools for collaboration and the communication of complex ideas \cite{Cantoni2009}. Such empirical techniques could assist in the discovery of unexpected values through defamiliarization \cite{LeDantec2009}, or be utilised to communicate the functionality of (and critique the values embedded into) the designs of system infrastructure such as algorithms: where their complexity frequently leads to practical opacity, impeding productive critical analysis and inviting speculation through external channels \cite{lee2015}.

\subsubsection{Mitigate Power Differentials Between Stakeholders}

The inclusion of stakeholders and the application of values levers doesn't necessarily mean that a produced design will be worker-centred, however: as seen in this project, critical discussions and an awareness of potential issues may not always be enough to produce end designs which satisfy all parties. We posit that in this study, this was partly due to one party having the majority of the control over the design and implementation of the platform, accentuating the representation of their values and agendas within the final design over those of the other stakeholders. We therefore recommend that measures are taken during design engagements to not only ensure that all stakeholder voices are heard, but that there is a continuing process of communication and accountability in situations of unequal distributions of power between stakeholders, requiring that all are kept abreast of developments and how issues are being addressed on a continuing basis by those with greater control over the design's implementation.
\section{Conclusion}

This paper has reported on the findings of a design ethnography undertaken with an international NGO and a for-profit gig-economy company in Bangladesh, during the design of an IVR component for use with a disadvantaged workforce of domestic workers. We argue that these design discussions regarding the IVR system acted as a `values lever': that the limited nature of the IVR format introduced a number of constraints which required the stakeholders to make design decisions which prioritised particular qualities within the system, and that discussing these decisions surfaced their values in relation to the project and their assumptions about the workers. We offer suggestions for how to introduce such values levers in technology production-focused contexts, arguing that the introduction of constraints could be a useful technique for values reflection: particularly in projects where parties may have different---even conflicting---agendas. As such a relationship is seemingly almost guaranteed between for-profit gig-economy companies and their employees, we argue that such steps should be considered during the production of any platforms looking to develop a worker-centred perspective.

\section{Acknowledgments}
%% The acknowledgments section is defined using the "acks" environment
%% (and NOT an unnumbered section). This ensures the proper
%% identification of the section in the article metadata, and the
%% consistent spelling of the heading.
\begin{acks}
Anonymised for review
\end{acks}

%%
%% The next two lines define the bibliography style to be used, and
%% the bibliography file.
\bibliographystyle{ACM-Reference-Format}
\bibliography{references}


\end{document}
\endinput

