\section{Related Work}

\subsection{The Gig Economy \& South Asia}

The `gig economy' tends to refer to software platforms which facilitate people selling their labour on demand, frequently (but not always) as a form of supplementary income on top of a regular job \cite{Taylor2017}. To lower costs and support scalability, these platforms tend to place a focus on the allocation of `most suitable' workers per job by a set algorithm \cite{Wood2019}. This lack of overhead means that gig work platforms can offer a high level of flexibility and autonomy, as workers can ostensibly have control over when and where they work---qualities which have made such platforms particularly attractive to young people \cite{Wood2019, Balaram2017, carlos2021}. The success of ride-sharing and delivery services such as Uber has meant that the `gig' model is being explored in other service industries, such as retail \cite{Balaram2017}. 

However, the recent popularity of gig work has attracted both caution and criticism. The industry's focus on quantified worker ratings and algorithmic assignment has been shown to result in low pay, social isolation and overwork \cite{Wood2019}. Questions have also been raised around current employment laws, and if they are capable of addressing workers' needs---such as sickness protection---when applied outside of traditional employment models \cite{Taylor2017}. Critics argue that the innovation and `disruption' lauded by the industry's benefactors do not inherently uplift its workers, due to fears of exploitation through a `race to the bottom' of cheap pricing and low-cost labour \cite{Balaram2017}.

Nevertheless, gig work has remained popular, particularly in South Asia: India and Bangladesh are some of the fastest growing freelancer markets \cite{Payoneer2019}. Ride sharing services are increasingly popular within Bangladesh, especially amongst tech-literate students and workers enticed by flexible work hours \cite{islam2019}, and their growth has encouraged tens of thousands of rural and suburban youths to migrate to metropolitan areas \cite{Fairwork2021}. This rapid and underregulated growth has prompted groups such as FairWork to call for government and social action \cite{Fairwork2021}, highlighting that in Bangladesh around 70\% of this new market is controlled by just two companies (Uber and Pathao) \cite{islam2019}. Ahmed argues that those in Bangladesh without access to digital technologies cannot access these new employment opportunities, deepening the digital divide, and highlights issues around workers being underpaid, overworked and constantly monitored \cite{Ahmed2020}. Responses to FairWork's call for government action may come too late for many workers: Minter argues that while there is a need for governments to introduce enforceable labour standards, these will take time to introduce; she suggests that in the short-term, non-state actors should work with gig-economy companies on formal agreements to support workers' fair treatment while government solutions are being negotiated \cite{Minter2017}. This paper engages within this suggested context of formalised NGO interventions in the gig economy.


\subsection{The role of HCI in the Gig Economy }

The HCI research community's involvement in the gig economy can be traced back to its undiluted form, `crowdworking', where workers on platforms such as Amazon's Mechanical Turk and Crowdflower are given microtasks and paid per acceptable completion. Such platforms have been previously used as cheap and easily accessible sources of research participants (e.g. \cite{mason2012conducting, mcnaney2016, Othman2017}). Jacques argues that while academic such discourse frequently focused on improving such platforms' efficiencies, little investigation was taken into the workers themselves: their socio-economic status and the impact of the platforms' designs on them \cite{Jacques2019}.

Recent years have seen more critical analyses: Martin et al. note the dehumanising rhetoric surrounding crowd workers (e.g. `artificial artificial intelligence', `cogs in the machine'), and how such terminology makes them easier to regard as `troublesome components' to be controlled, rather than real human stakeholders worthy of design considerations \cite{martin2016}. While researchers often assumed that crowdworkers did microtasks for fun or pocket money \cite{martin2016}, Hara et al. argue that despite 96\% of Mechanical Turk workers earning less than the US federal minimum wage, the primary goal of most is income generation \cite{Hara2018}. Furthermore, they note that many are from groups traditionally excluded from formal labour markets (e.g. people with disabilities, autistic people), suggesting that such platforms contribute to the exploitation of vulnerable groups.

A common frustration relating to gig economy platforms is a lack of transparency: Martin et al. argue that the deeper functioning of gig economy systems (e.g. the specifics of work assignment algorithms) is often opaque to the worker, leading to worker frustration and a balance of power in favour of clients and platform holders \cite{martin2016}. Furthermore, gig economy platforms are frequently not designed to support communications between workers, which has been identified as one factor limiting gig worker collective bargaining \cite{Hara2018}. This weak bargaining power leads to the pace of work being determined by direct demands from clients, heightened by a lack of job security and a frequent oversupply of labour \cite{Wood2019}. Lee et al. argue that increased transparency in the assignment process could elicit greater cooperation with work assignments, especially undesirable ones: because the current supply-demand control algorithms do not account for human factors (such as workers' capability and motivations), their use in motivating and controlling human behaviors created distrust of the system in workers \cite{lee2015}. Raval \& Dourish note that gig platforms' monetisation models frequently erase the distinction of work and `related work' (such as care labour), and that platforms place workers' own bodies and possessions as the sites of engagement between clients and corporations: placing additional focus on workers' emotional performance, bodily presence and timeliness \cite{raval2016}. This combination of opaque, quantified evaluation and an apparent accountability for every interaction creates a hyper-awareness of clients' ratings and the potential for psychological stress \cite{lee2015}.

While potential solutions to these issues are worth exploring, introducing technology interventions without care runs the risk of reifying pre-existing inequalities amongst workers, or even creating new power dynamics within a given platform's economy \cite{martin2016}. In response to these issues, Alvarez et al. call for a greater worker-centred perspective in the design of gig economy platforms, focusing on transparency, professional development, networking and an avoidance of power asymmetry \cite{carlos2021}. However, Lee et al. highlight that when compared to designing and building traditional user interfaces, designing the algorithms used by gig economy platforms to manage and assign workers in a human-centred approach will require different methods of specifying and evaluating stakeholder requirements, states, and interactivity \cite{lee2015}.


\subsection{Design methods and tools to surface values in design }

Review of this

The gap here is (i) few examples and (ii) tend to be in research contexts rather than applied contexts. Often in applied contexts critical research isn’t carried out. 
