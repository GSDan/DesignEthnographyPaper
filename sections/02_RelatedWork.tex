\section{Related Work}

\subsection{The gig economy }

technologically facilitated ondemand labor 

Gig economy tends to refer to people who use apps to sell their labour, often supplementary income on top of a regular job \cite{Taylor2017}

attractive to young people, looking to expand to other areas such as retail \cite{Balaram2017}

their operation places a focus on algorithmic control, which while offering workers a high level of flexibility (log on when you want) and autonomy have also been shown to result in low pay, social isolation and overwork \cite{Wood2019}.

this new way of working raises questions about the suitability of the current employment law framework in addressing the needs of people actively choosing to work outside of the traditional employment model (e.g. sickness protection) \cite{Taylor2017}

fears of platforms being exploitative, danger of a race to the bottom, innovation and `disruption' not inherently uplifting for workers \cite{Balaram2017}

South Asia, particularly the Philippines, India and Bangladesh one some of the fastest growing freelancer markets \cite{Payoneer2019}

ride sharing services increasingly popular, especially amongst students as they're well acquainted with digital devices. Majority of BD rideshare market controlled by Uber and Pathao \cite{islam2019}

People in Bangladesh, embraced this technological platform for its economic benefits and flexible work hours. [gig economy] ecosystem should be extended to include people who are not technologically adept but can still contribute their physical labor to the gig economy platforms
that offer delivery and ride sharing services. the issues of ethical labor relations should also be addressed so that these workers are not underpaid, overworked, and constantly monitored \cite{Ahmed2020}

Fairwork BD - rapid growth of the gig economy in Bangladesh has encouraged tens of thousands of rural and suburban youths to migrate to metropolitan areas to take part in this transformation. calling for government and social action \cite{Fairwork2021}

Need for enforceable labour standards, in short term there's a need for non-state actors to work with gig-economy companies on formal agreements \cite{Minter2017}




\subsection{The role of HCI in the Gig Economy }

crowdsourcing - e.g. mechanical turk
As academic discourse has focused on improving crowdsourcing models to reduce costs and increase efficiency, the socio-economic status of the workforce and the impact of these models in a expanding market of what might otherwise be undertaken as “at-will” employment has been the subject of little direct investigation \cite{Jacques2019}

HCI has used platforms like Amazon Mechanical Turk as a resource for cheap research participants \cite{mason2012conducting, mcnaney2016, Othman2017}

shift towards taking a more critical eye

Gig workers dehumanised: referred to via machine metaphors 'artificial artificial intelligence,’ ‘cogs in the machine'; by not thinking about them as ‘real’ human beings with needs, problems and troubles, it was easy to consider them as troublesome ‘components’, needing to be controlled, and not worthy of the usual design considerations extended to other stakeholders. It was also easy to make up stories on their behalf such as they do this for fun. \cite{martin2016}

general assumption that many gig-economy workers (especially mechanical turk) do this work for extra cash or as a hobby, whereas a considerable number of workers, particularly in south asian countries like india, rely upon it for income. [cite - check hara2018?]


focus on the experiences of workers outside engaging with gig-economy platforms outside of the actual work itself and how these structures can inform the balance of power between workers, clients and platforms. e.g. Underlying functions of platforms can be largely obscured, leaving the deeper functioning of the marketplace opaque to the workers. Puts balance of power in the hands of the platform and the clients.  Many gig-economy workers are not part of an organisational structure with rights and responsibilities - cooperation is not part of the job, so collaboration happens off-platform to share information (e.g. identify and avoid bad clients)  \cite{martin2016} \cite{lee2015}

platforms monetisation patterns frequently erase the distinction of work and related work, such as taxi drivers' care labour and the removal of tips; gig platforms place individual workers' bodies and their own possessions as  the sites of engagement between clients and corporations, placing direct focus on each worker's emotional performance, bodily presence and timeliness  \cite{raval2016} Being held accountable for every interaction, drivers were very aware of the existence of being rated by clients. Trying to deliver good services for all service interactions could pose psychological stress to workers.\cite{lee2015}

96\% of workers on Amazon Mechanical Turk earn below the U.S federal minimum wage, platforms not being designed to support comms between workers and between workers and crowdwork `requesters' identified as an issue, limiting collective bargaining \cite{Hara2018}. 

hara and martin both highlight the importance of technology in information flow within gig platforms, and how access to this information is strongly linked to power. use of technology to support worker collaboration to help them avoid bad employers.

transparency of assignment process could elicit greater cooperation with assignments, especially undesirable ones. Supply-demand control algorithms were originally designed to solve mathematical optimization problems that involve non-human entities. In Uber and Lyft, however, they are used to motivate and control human behaviors - lack of account of human factors (capability, motivations) combined with lack of transparency created distrust of the system in workers  \cite{lee2015}

Martin warns of danger of technology interventions reifying pre-existing inequalities amongst workers or create new winners and losers \cite{martin2016} 



Embedded socio-ethical issues. E.g. who this technology includes / excluded through tech access, capacity of the system, data literacy, coded bias and values. 

A need to surface values and design more transparently. (brief review of CHI research and others who have looked at embedded values and access etc through the design process).  


this research raises the need for new methodological research in HCI and interaction design on designing human-centered algorithmic management. HCI and interaction design have established systematic processes and methods for designing human-centered interfaces and interactions. Compared to designing and building traditional user interfaces, designing algorithmic management will require different ways of specifying and evaluating requirements, states, and
interactivity



\subsection{Design methods and tools to surface values in design }

Review of this

The gap here is (i) few examples and (ii) tend to be in research contexts rather than applied contexts. Often in applied contexts critical research isn’t carried out. 

\subsection{Introduce our research}

In this paper, we contribute to these discussions by describing an applied partnership between Ox, HT and Mo that aimed to support a gig economy system for domestic workers in BD. The premise of the project was to empower the workers by increasing access to work.

Through this process we introduce the design and use of an IVR system to promote comms between clients and workers to cater to tech access of workers. However, the process of designing the IVR system presented a series of constraints the forced values / priorities of various partners to be surfaced and negotiated and worked through. 

Through this process there are two contributions: (i) a case study example of the design of a gig economy information infrastructure in a diverse tech context; and (ii) how the tech of focus (IVR system) through its constraints became a design material to surface differences in motivations for the working of the system, and ways to negotiate these for HCI researchers to consider in future.  
