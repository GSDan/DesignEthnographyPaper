\section{Findings}

\begin{figure*}
  \centering
  \includesvg[width=1.5\columnwidth]{images/hellotask_diagram.svg}
  \caption{The \PC{} IVR system, as encountered by a domestic worker. (1) The system calls the worker in the morning to ascertain if they want to work, and if so when. (2) The client requests a worker through the smartphone app. \PC{}'s algorithm chooses a suitable worker and rings them to see if they will take the order. If so, the worker is asked to wait while \PC{} check if the order is genuine. The worker receives a second confirmation call from the system, and if the job is genuine is then asked to give an ETA and leave towards the client's house. (3) During the journey, the worker is called up three times to get an updated ETA. (4) At the client's house, the worker calls the system to log when the work is started and finished. (5) After finishing, the worker waits at the client's house until an automated SMS arrives confirming that the client has paid. At any time, the worker can call the system to get information about a current job or their wages, get help from the \PC{} office or emergency services, or to contact the client. }~\label{fig:WorkFlow}
\end{figure*}

The following findings are presented in relation to the final \PC{} system design from the perspective of the domestic workers: ways in which they directly engage with the IVR component of the system during a single `job cycle', from giving notice of availability to the final sign-off after a job is complete (illustrated in Figure \ref{fig:WorkFlow}). We outline the purpose and quality of the interactions between the domestic workers and the IVR system as they are designed, drawing on recordings and notes from discussions held between between the research team, \PC{}, and \NGO{}.

\subsection{Daily Check of Availability}

Each domestic worker registered to \PC{} receives an automated call from the IVR each morning to determine whether they are available to work that day, and if so are they are available to work in the morning or afternoon. This process was devised by \PC{} and \NGO{} to give all domestic workers registered to the platform an equal opportunity to work within their local area during a time that suited them. Workers' responses to the IVR are logged onto the \PC{} system, informing an automated worker selection algorithm of which workers are available throughout the day. 

\subsection{Being Offered a Job}

During their nominated working time, workers will then receive another automated phone call through the IVR system, saying:

\begin{displayquote}
\textit{You have a work request of [X] hours from [area name] and it will pay you [Y] taka. The user's name is [name]. You [have/haven't] worked there previously. Press 1 to accept the order. Press 2 to reject the order. Press 3 to hear the order again.}
\end{displayquote}

If they accept, they are played an audio recording detailing the exact location of the client's home, and asked to either reject the job or re-confirm it by giving an estimated time of arrival (ETA), with the options of 30 minutes, 60 minutes, or longer than 60 minutes. There is no option to delay or negotiate an alternative time. If the worker rejects the order or if they hang up, the system will call the next worker prioritised by the algorithm.

\subsubsection{Prioritising Workers for a Job}
The group's discussions around this IVR menu prompted queries about how the platform's algorithm should select workers. \PC{} and \NGO{} negotiated three factors that were aligned with their priorities: 

\begin{itemize}
  \item \textit{Proximity of the Worker's Home to the Client}: the algorithm prioritises workers within a 1km radius of the client’s home, and then gradually extends out to 2km depending on availability. This reduces the travel time and cost to the worker, whilst also increasing worker reliability as they are more likely to arrive at their estimated time. 
  \item \textit{Work History}: workers who have previously worked for a client will get priority with them, unless they received a poor rating from the client. This was to promote stability for both workers and clients.
  \item \textit{Ratings of the Worker}: workers are given two quantified scores, reflecting i) the frequency in which they accept or reject offers of work and ii) their average rating given by past clients (out of 6).
\end{itemize}

However, these three factors present in the final design are not inclusive of all concerns brought up during these discussions. For example, \NGO{}'s representative suggested that the algorithm should take workers' ratings of clients into account---a feature absent from the `Human IVR' implementation:

\begin{displayquote}
"One rating will be from the employer’s side: if she gets five [stars] she’ll be called again. We are also taking ratings from the employee level---how she feels working there. Are you going to consider this? Because it might be that she might not be comfortable with working there: it might be that I’m a bad person, and give her five stars to get her again, but she’s not comfortable with me."
\end{displayquote}

Further discussions centred around the fact that the worker was given little access to information regarding both the client and the job when choosing whether or not to accept. For example, the IVR menu design did not give the worker information regarding the nature of the work (i.e. what they would be doing), the client's rating by other workers, or who will be present in the household during the work. In contrast, clients were given access to the worker's past ratings, a profile photo, the amount of training they had received, and their phone number. Some of these omissions could be explained by the limitations of IVR, the need for timeliness, and that logistical information about the job took priority. However, \PC{} were also concerned that the inclusion of additional information such as client ratings could `confuse' the workers---when we suggested that workers be provided with clients' ratings prior to accepting a job, \PC{} responded by saying:

\begin{displayquote}
"The domestic worker might not understand the rating system to be able to assess the client. So I think we need to do the screening from the back end."
\end{displayquote}

When we raised the potential safety and privacy issues regarding giving clients access to female workers' phone numbers \cite{Sambasivan2019}, \PC{} claimed that it hadn't been an issue. However, as removing access to phone numbers could potentially lower the worker `bounce rate' (when clients use the workers' details to hire them directly, outside of the \PC{} platform) they became interested in routing the calls through the IVR as an intermediary. At the time of writing, the \PC{} app still gives clients access to the workers' phone numbers. 

\subsection{Confirming the Job}

After the worker accepts the job request, they are asked to wait while a \PC{} customer support agent manually calls the client directly to confirm that the job is legitimate: \PC{} explained that clients often place test orders to see how the system works, and so they call to confirm every order. When we queried why the domestic workers were called prior to confirming with the client that the job is legitimate, \PC{} expressed that this process meant they could avoid cancelling on a client in an instance where they couldn't find a suitable worker:

\begin{displayquote}
"We do not want to call the user without first confirming that we have a domestic worker in place to serve him. It will make him dissatisfied." 
\end{displayquote}

If the client confirms the job, the worker then receives a second automated call saying that the job is legitimate, and reminding the worker of a number of safety procedures: 

\begin{displayquote}
\textit{Your order has been confirmed. Please set out for [name]’s house at [area]. Please only enter the user’s house if you see a woman in the house, wear a mask, and immediately report any safeguard issue by calling us. Call us back at [phone] whenever you need any support. To hear the user's detailed address, press 1. To talk to the user, press 2. To talk to the office, press 0. Otherwise, please hang up now and set off for the user’s house.}
\end{displayquote}

The \NGO{} representative noted that the safety checks included here were part of a broader strategy devised by \NGO{} to explicate the importance of safety, and aimed to exclude particularly `risky' client groups from the service:

\begin{displayquote}
"We do not provide the service to any single male households---bachelors. We normally provide the service to the family. [It is] compulsory that there is one female member present during the work day."
\end{displayquote}

\subsection{Travelling to the Job}
Once a job is confirmed, the IVR system calls the worker as they travel to the client's home up to three times to `track’ their progress and get an updated ETA. The first call is made 15 minutes after the confirmation call, and asks if the worker has left their home yet; the second call is made after two thirds of the given ETA has expired, and asks for an updated ETA; and the final call is made 10 minutes after the ETA if the worker has not yet logged the start of work. These calls offer workers the option to connect to the client for a single call per job, to support workers asking for more specific directions without irritating the client. 

When we queried about why workers were being called on three occasions over a short period of time and suggested that it might be intrusive, \PC{} expressed that they had doubts about worker reliability and professionalism, expecting this number of calls to be necessary:

\begin{displayquote}
"We have to maintain a certain time---on the user’s side we have already committed them at the start of the order that it will take 30 minutes, based on the estimation of the domestic worker [...] so we keep pushing them. [...] When we test the system we might find that we need more tracking calls, maybe we don’t need this many, even."
\end{displayquote}

\PC{} received negative feedback from their workers about the tracking calls during the `Human IVR' prototype. Despite finding that they were unnecessary during the prototype, they expected the calls would still be needed in the final automated version:

\begin{displayquote}
"Maybe they’re on their way, on the bus---they really don’t want to receive a lot of calls. [...] We don’t need to call that much: without calling the domestic workers, after half of the time almost all of them went to the customer’s house. [...] But I’m worried that in the case of the IVR, it will be lower: it may be that we need at least one tracking call at the start [of the journey]."
\end{displayquote}

\subsection{Safety Measures \& Starting the Job}

Once the worker arrives at the client’s home, they call into the IVR system to log the start of work. Prior to doing so, they are required to check that the client has met the safety conditions of the platform: ensuring there is at least one woman in the home, and that the worker feels comfortable and safe. If these safety requirements aren't met, \PC{} explained that the workers are advised to leave the house immediately and call the IVR to log a safety breach: 

\begin{displayquote}
"Our domestic workers are instructed to check if there is a female member or not. What happens if they see the house doesn't have a female member? They come back. We have a status called ‘maid return’. It happens, maybe every day two or three times."
\end{displayquote}

Should these safety measures not be sufficient, workers can call into the IVR system and are given options to connect with emergency services and an independent women's helpline. After feedback from the research team, \PC{} moved these options to be read at the beginning of the IVR menu, ahead of more frequently used options relating to starting and finishing a job or checking on earnings. 

\PC{} and \NGO{} emphasised that prior to the IVR system, logging the start and end of a job used to be in the control of the client through the smartphone app, but that they would sometimes deliberately log inaccurate times to reduce the cost of work:

\begin{displayquote}
"Sometimes the customers are very tricky---he will not say anything [about the time being over] because she is busy doing something."
\end{displayquote}

\subsection{Completing the Job \& Payment}

Once the worker completes a job, they are required to call into the IVR system and log the end of work from the client’s home. The client then makes an online payment via the \PC{} app. Once the payment is complete, the worker is then sent an SMS stating:

\begin{displayquote}
\textit{The client has paid the bill. You can leave the house. Your payment will arrive in your account shortly.}
\end{displayquote}

However, when we queried if the often illiterate workers are able to understand the message, \PC{} responded:

\begin{displayquote}
"No, not really. Sometimes they can use it as an indication---there is a sound."
\end{displayquote}

In our discussions, \PC{} emphasised that workers are permitted to leave the client’s home only after they have received this confirmation message. \PC{} explained that cash payments were no longer permitted, as online payments enabled them to regulate payments more transparently:

\begin{displayquote}
“If the workers are paid in cash like in their other jobs, they can sometimes not get paid or get paid less”. 
\end{displayquote}


\subsection{Leaving Feedback about the Client}

After leaving the client’s home, the workers have the option to leave feedback about their experience with that client by calling the IVR system, which gives the option to rate their client out of six stars or give additional feedback by leaving a recorded message. These messages are later listened to by a \PC{} staff member and logged onto the system. If a worker reports a negative experience with a client, \PC{} addresses this by calling a worker directly and clarifying the issue, and---if deemed necessary---will call the client to inform them of a breach of conduct and potentially block them from using the service for a period of time. 