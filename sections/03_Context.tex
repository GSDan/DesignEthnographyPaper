\section{Context}

During this project we (members of an HCI research group based in Australia) entered into a three-way partnership with the Bangladeshi branch of an international NGO (anonymised as `\NGO{}') and \PC{} (anonymised), a gig-economy startup company based in Dhaka, Bangladesh (BD). Full ethical approval was received from our institutional review board before work commenced.

\NGO{} was in the midst of a project aiming to improve the well-being of female domestic workers in BD, who are typically hired to cook, clean, do laundry or even care for the children of a household. Estimates range between there being 2 million \cite{DWRN2011} and 4 million \cite{Ashraf2019} domestic workers in BD, with around 80\% being women or girls \cite{Ashraf2019}. Despite their prevalence, domestic work in BD was only recognised as an informal profession in 2015, with the introduction of the Domestic Workers Protection and Welfare Policy (DWPWP) \cite{Islam2017}. Prior to this, domestic workers were not entitled to time off, were not legally assured `fair' wages (85\% live under the poverty line \cite{BILS2015}), and there were few legal protections from abuse and harassment within their places of work \cite{IDWF2015}. However, adherence to these new policies has been inconsistent \cite{islam2016}, with a perceived lack of policy enforcement and reports of abuse still frequent \cite{DailyStar2018}. Within this context, \NGO{} aims to provide women domestic workers with skills training for formal job opportunities, to increase their awareness of their rights, and to support the BD government's capacity to enforce and monitor the implementation of the DWPWP. 

As discussed, gig-economy platforms are increasingly common in BD \cite{Ahmed2020}. In-line with the prevalent trend of technology-focused social interventions within BD (e.g. \cite{hasnayen2016, Faroqi2019}), \NGO{} chose this as a sector through which to promote their agenda and partnered with the gig-economy startup \PC{}. Aiming to be `the Uber of domestic workers', \PC{} runs an app-based service through which customers can request a domestic worker using their smartphone. In exchange for configuring their business towards the promotion of the DWPWP and the empowerment of domestic workers, \NGO{} are assisting \PC{} with the recruitment and training of several thousand domestic workers to support higher rates of pay. On \PC{}'s app, the `About Us' page notes that they `\textit{dream to build an ecosystem where every family employs trained, skilled and verified domestic helpers}' and that they aim to provide a `\textit{secure workplace for millions of domestic helpers through our platform and establish "domestic work" as a dignified profession}'. As such, \PC{}'s priority was to create a platform which would enable them to: i) meet commercial interests, and ii) empower domestic workers.

\PC{}'s employed domestic workers, like many women in BD, live in a digital divide driven by gender and socio-economic status \cite{Genilo2015}. Most lack access to smartphones, usually only having a `feature phone' which can place calls and send/receive SMS (although many have limited textual literacy, rendering SMS of limited value). As a result, the majority of \PC{}'s workforce are unable to directly interact with the digital infrastructure to respond to client requests. In response, \PC{} initially introduced `local guides': another tier of gig-economy workers who would connect customers to workers by taking orders from clients through the PC app and forwarding their details to domestic workers through phone calls. However, after several months \PC{} reflected that this process was costly and biased: reporting that local guides would often select workers based on personal preferences, leaving the process open to misuse. 

In late 2020, \PC{} and \NGO{} decided to explore alternative, automated solutions to bridge the domestic workers with the digital infrastructure. They decided to create an Interactive Voice Response (IVR) system: an automated telephone-based system which can be interacted with by callers through button presses. An automated IVR system was deemed to be more efficient and objective, as it provided a simple way to connect clients to users through an automated decision-making process that was thought to be less influenced by human bias. As a research team with experience of designing IVR systems and a pre-existing relationship with \NGO{}, we offered consultation and advice during the design process of this new system - whilst simultaneously carrying out a design ethnography of the process to reflect on how the values of the three parties were reflected in the design of the IVR system. 