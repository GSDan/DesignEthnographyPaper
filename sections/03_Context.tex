\section{Context}

During this project we (members of an HCI research group based in Australia) entered into a three-way partnership with the Bangladeshi branch of an international NGO (anonymised as `\NGO{}') and \PC{} (anonymised), a gig economy startup based in Dhaka, Bangladesh. Full ethical approval was received from our institutional review board before work commenced.

\NGO{} was one of a number of organisations involved in a multi-million dollar international project (`\SRP{}', anonymised) aiming to improve the safety and well-being of Bangladeshi female domestic workers. Domestic workers are typically hired to cook, clean, do laundry or even care for the children of a household. Estimates range between there being 2 million \cite{DWRN2011} and 4 million \cite{Ashraf2019} domestic workers in Bangladesh, with around 80\% being women or girls \cite{Ashraf2019}. Despite their prevalence, domestic work in Bangladesh was only recognised as an informal profession in 2015, with the introduction of the Domestic Workers Protection and Welfare Policy (DWPWP) \cite{Islam2017}. Prior to this, domestic workers were not entitled to time off, were not legally assured `fair' wages (85\% live under the poverty line \cite{BILS2015}), and there were few legal protections from abuse and harassment within their places of work \cite{IDWF2015}. However, adherence to these new policies has been inconsistent \cite{islam2016}, with a perceived lack of regulatory enforcement and reports of abuse still frequent \cite{DailyStar2018}. Within this context, the \SRP{} project aims to provide female domestic workers with skills training for formal job opportunities, to increase their awareness of their rights, and to support the Bangladesh government's capacity to enforce and monitor the implementation of the DWPWP. 

In line with the policies promoting technologically-mediated social interventions in Bangladesh (e.g. \cite{hasnayen2016, Faroqi2019}), \NGO{} chose to partner with a gig economy company \PC{} as a part of the \SRP{} project. As discussed, gig economy platforms are increasingly common in Bangladesh \cite{Ahmed2020}. Aiming to be `the Uber of domestic workers', \PC{} runs an app-based service through which customers can request a domestic worker through a smartphone app. Whereas other organisations within the \SRP{} project recruited and provided training to domestic workers, \PC{} provided work for them: configuring their business towards the promotion of the DWPWP by highlighting the workers' rights and implementing platform policies to improve the women's working conditions. On \PC{}'s app, the `About Us' page notes that they `\textit{dream to build an ecosystem where every family employs trained, skilled and verified domestic helpers}' and that they aim to provide a `\textit{secure workplace for millions of domestic helpers through our platform and establish "domestic work" as a dignified profession}'. Taken at face value, \PC{}'s priorities are to create a platform which would enable them to: i) achieve their commercial goals, and ii) empower domestic workers.

Like many women in Bangladesh, \PC{}'s domestic workers are subject to a digital divide for which gender and socio-economic status are defining characteristics \cite{Genilo2015}. Most lack access to a smartphone, typically only having a `feature phone' which can place calls and send/receive SMS (although low literacy levels render SMS of limited value). As a result, the majority of \PC{}'s workforce are unable to directly interact with the digital infrastructure to respond to client requests. Because of this, \PC{} initially utilised `local guides'---another tier of gig economy workers---whose function was to connect clients to workers by taking orders through the \PC{} app and forwarding their details to domestic workers using phone calls. However, after several months \PC{} reflected that local guides would often select workers based primarily on personal preferences, leaving the process open to favouritism. 

In late 2020, \PC{} and \NGO{} decided to explore alternative, automated solutions to bridge the domestic workers with the digital infrastructure. They decided to create an Interactive Voice Response (IVR) system: an automated telephone-based system with which callers could interact using phone button presses. An automated IVR system was deemed to be more efficient and objective, as it provided a simple way to assign workers to clients through a consistent, automated decision-making process that would address previous concerns relating to the local guides. As a research team with experience of designing IVR systems and a pre-existing relationship with \NGO{}, we offered consultation and advice during the design process of this new system, whilst simultaneously carrying out a design ethnography of the process to reflect on how the values of the three parties (\NGO{}, \PC{} and the research team) were reflected in the design of the IVR system. 