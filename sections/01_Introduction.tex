\section{Introduction}
It is essential that designers and engineers have awareness of how their assumptions and priorities are embedded into technology platforms, as such design decisions have the potential to both benefit and hinder a system's stakeholders \cite{winner1980}. The HCI community has introduced and utilised a wide variety of frameworks and methods to surface and critically engage with the values of a system's stakeholders (including its designers), however these are primarily intended for use by HCI researchers: requiring significant time commitments, an understanding of empirical research methods, and sometimes even reviews of philosophical literature \cite{friedman2006, flanagan2014, Alshehri2020, shilton2018}. As such, these methods are unlikely to be adopted by small-medium production-focused organisations who often lack dedicated UX designers and research staff \cite{Shilton2013, Ardito2014}. While certain informal practices (dubbed `values levers') have been shown to serve as effective entry points for value discussions within production-focused settings \cite{Shilton2013, shilton2018, shilton2019}, little actionable guidance has so far been provided as to how they can be purposefully and effectively deployed within industry.

This paper works within this space to highlight the particular importance of producing value-centred technologies within the `gig economy': the use of online platforms to connect customers to workers, an industry which has seen a worldwide explosion in popularity in recent years \cite{Taylor2017, islam2019, Wood2019}. The technologies that power the gig economy need to account for and balance the disparate needs of two primary stakeholder groups: their customers (and, in turn, company profit), and their workers (whose requirements may vary, based on context and field of work \cite{carlos2021}). The pace at which the gig economy has grown and `disrupted' markets has presented its own issues and employment law has struggled to keep up \cite{Minter2017}, resulting in fears of worker exploitation due to a lack of necessary protections and safety nets \cite{Balaram2017}. While waiting for official legislation and regulation of the gig economy to be introduced, there have been calls for non-state actors---such as unions and NGOs---to negotiate formal agreements with gig economy companies, providing short-term benefits to workers and highlighting current issues \cite{Minter2017}. 

This paper directly engages within this context, and presents findings from an extended series of collaborative design engagements with an international advocacy group (pseudonym: `\NGO{}') and a start-up company (`\PC{}'), which runs a gig economy platform for disadvantaged domestic workers in Dhaka, Bangladesh. The product of these engagements is an Interactive Voice Response (IVR) system, designed to connect the predominantly offline domestic workers to \PC{}'s digital infrastructure in order to distribute and manage jobs. Using this system as a lens, this paper contributes: (i) the findings of a design ethnography, documenting how the creators' values, motivations and assumptions are embedded within the system; (ii) a demonstration of how the constraints of a technology medium can act as a values lever for critical reflection; (iii) discussion for how constraint-based values levers can support critical reflection in resource-constrained commercial development contexts.