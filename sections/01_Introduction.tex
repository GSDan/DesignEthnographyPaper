\section{Introduction}
It is essential that designers and engineers are aware of how their assumptions and priorities are embedded into technology platforms, as such design decisions have the potential to both benefit and hinder a system's stakeholders \cite{winner1980}. The HCI community has introduced and utilised a wide variety of frameworks and methods to surface and critically engage with the values of a system's stakeholders (including its designers) \cite{friedman2006, flanagan2014, Alshehri2020, shilton2018}, however these tend to be intended for use by HCI researchers: requiring significant time commitments, an understanding of empirical research methods, and sometimes even reviews of philosophical literature. As such, these methods are likely to be unattractive to small-medium production-focused organisations who often lack dedicated UX designers, nevermind research staff \cite{Shilton2013, Ardito2014}. While certain informal practices (dubbed `values levers') have been shown to serve as effective entry points for value discussions within production-focused settings \cite{Shilton2013, shilton2018, shilton2019}, little actionable guidance has so far been provided as to how they can be purposefully and effectively deployed within industry.

This paper works within this space to highlight the particular importance of producing value-centred technologies within the `gig economy': the use of online platforms to find small jobs, an industry which has seen a worldwide explosion in popularity in recent years \cite{Taylor2017, islam2019, Wood2019}. The technologies which power the gig-economy need to account and balance the disparate needs of two primary stakeholder groups: their customers (and, in turn, company profit), and their workers (whose requirements may vary, based on context and field of work \cite{carlos2021}). The pace at which the gig-economy has grown and `disrupted' markets has presented its own issues and employment law has struggled to keep up \cite{Minter2017}, resulting in fears of worker exploitation due to a lack of necessary protections and safety nets \cite{Balaram2017}. While waiting for official legislation and regulation of the gig-economy to be introduced, there have been calls for non-state actors---such as unions and NGOs---to negotiate formal agreements with gig-economy companies, providing short-term benefits to workers and highlighting current issues \cite{Minter2017}. 

This paper directly engages within this context, and presents findings from an extended series of engagements from collaborative design engagements with an international advocacy group (pseudonym: `\NGO{}') and a start-up company (`\PC{}'), which runs a gig-economy platform for disadvantaged domestic workers in Dhaka, Bangladesh. The product of these engagements is an Interactive Voice Response (IVR) system, designed to connect the predominantly offline domestic workers to \PC{}'s digital infrastructure in order to distribute and manage jobs. We present the results of a design ethnography undertaken during this process, discussing the system's design and how both \NGO{} and \PC{}'s values, motivations and assumptions have been embedded within it. We contribute: our insights from this series of design engagements; discussions of how the restrictions of the IVR medium acted as a `values lever', surfacing and supporting critical engagement with the values, priorities and assumptions held by the system stakeholders; and offer suggestions for how this can be done during the production of real-world, commercial systems through the application of constrained design as a reflective design method.


