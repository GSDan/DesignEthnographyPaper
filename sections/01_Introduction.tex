\section{Introduction}

The use of online platforms to find small jobs has seen a worldwide explosion in popularity in recent years \cite{Taylor2017, islam2019, Wood2019}. Dubbed the `gig-economy', it exists as a labour-based subset of the `sharing economy', where workers capitalise on their skills and spare time as underused assets for income generation, often in addition to other commitments \cite{Balaram2017}. The pace at which the gig-economy has grown and `disrupted' markets has presented its own issues, however, and employment law has struggled to keep up \cite{Minter2017}, resulting in fears of worker exploitation due to a lack of necessary protections and safety nets \cite{Balaram2017}. However, any laws which do get introduced also need to protect the flexibility and autonomy offered to workers by these platforms, which made them appealing to so many in the first place \cite{Wood2019}. Similarly, the technologies which power the gig-economy need to account and balance the disparate needs of two primary stakeholder groups: their customers (and, in turn, company profit), and their workers (whose requirements may vary, based on context and field of work \cite{carlos2021}). An understanding of how designers' assumptions and priorities are embedded into these systems is therefore essential, as such design decisions have the potential to both benefit and hinder these stakeholders.

While waiting for this official legislation and regulation of the gig-economy to be introduced, there have been calls for non-state actors---such as unions and NGOs---to negotiate to formal agreements with gig-economy companies, providing short-term benefits to workers and highlighting issues with existing legislation \cite{Minter2017}. This paper engages within this context, and presents findings from an extended series of engagements from collaborative design engagements with an international advocacy group (\NGO{}) and a start-up company (\PC{}) which runs a gig-economy platform for disadvantaged domestic workers in Dhaka, Bangladesh. The product of these engagements is an Interactive Voice Response (IVR) system, designed to connect the predominantly offline domestic workers to \PC{}'s digital infrastructure in order to distribute and manage jobs. We present the results of a design ethnography undertaken during this process, discussing the system's design and how both party's values, motivations and assumptions have been embedded within it. We contribute our insights from this series of design engagements, suggestions for how designers and engineers can critically engage with their own values as a part of the design process, and discuss the potential pitfalls non-state actors face when attempting to formally advocate for workers while working within a framework of disempowerment.