\section{Discussion}

This IVR system enables domestic workers to participate in the gig economy: without requiring smartphones, internet access or any significant level of literacy. However, despite the project's primary goal being the empowerment of the domestic workers, there are a number of design decisions evident which would be detrimental to the workers' experiences when using the system. The design highlights that the IVR medium's inherent constraints required decisions to be made, resulting in some motives and values being prioritised over others throughout the workers' job cycle. We argue that when examined, these decision points act as values levers: revealing how the diverse and sometimes conflicting values of \PC{} and \NGO{} were prioritised and ultimately included (or excluded) from the final system design, often at the expense of the workers. In this discussion we reflect on how these values were surfaced and negotiated through the design, and discuss the broader implications of how constrained design can be used in commercial contexts as values levers. 

% NOTE TO PATRICK: THIS WAS PREVIOUSLY 6.2 AND HAS BITS OF THE PREVIOUS 6.1 IN IT

\subsection{Highlighting Embedded Values through Design Decisions as Values Levers}

Contrary to \PC{} and \NGO{}'s belief that the IVR system would offer more objectivity than the `local guides', such infrastructures feature embedded biases informed by the values and priorities of those involved in the design process \citep{winner1980}. In turn, analysing and querying the design of technological systems can give insights into the designer's priorities, attitudes and assumptions \citep{Star1999}. As such, querying the choices required during the design of the IVR component re-framed them as values levers: acting as entry points for reflection and values discussions \cite{Shilton2013}. In this way, we have been able to gain insights into how the stakeholders' values were embedded within the final system design, through the inclusion, prioritisation, and omission of pieces of information and functionality accessible to workers:

\begin{description}
    \item[\PC{}'s low expectations of the workers] There are multiple elements within the system's design which suggest that \PC{} had low expectations of the workers. For example, the `tracking calls' were perceived by \PC{} to be a necessary inclusion to prompt workers to arrive at clients' houses on time: even after a high success rate during the `Human IVR' deployment (where the calls had been removed due to complaints from the workers), \PC{} was concerned that workers would not leave on time if the calls were not coming from a human operator. As in crowdsourcing platforms, this intrusive use of technology for monitoring frames workers as `troublesome components' to be controlled, rather than human stakeholders to be designed for \cite{martin2016, Irani2013}. \PC{} also believed that the women would be confused if given the ratings of clients when offered work, and that such decisions should be made algorithmically `\textit{from the back-end}'. Furthermore, while clients had access to workers' phone numbers, workers were only able to directly connect to clients once by routing through the IVR system. These low expectations and the unequal provision of information and control contribute towards a final design where the worker has little power: as seen in analyses of other gig-economy systems \cite{martin2016, Hara2018, carlos2021, lee2015}, a lack of information flow and opaque, algorithmic control often results in a system where the worker has less power than other parties. 
    
    \item[Prioritising clients over workers] As noted, \PC{} are principally a for-profit company, and the system's design reflects their prioritisation of clients' interests over those of the workers. This is evident through design decisions both benign (e.g. the worker is called to accept a job before the client is called to confirm it is legitimate; that workers receive automated calls, while customers are called by human operators) and those that have serious potential consequences (e.g. that workers' ratings of clients were not provided during IVR calls offering jobs, nor being considered by the algorithm during `Human IVR' deployment; that \PC{} had few concerns related to safety when sharing workers' phone numbers with clients). As Lee argues, supply-demand orientated algorithmic controls frequently do not account for human factors \citep{lee2015}: such issues are further evident in this system, where workers are pushed through repeated phone calls during travel, prioritising keeping clients' waiting time to a minimum. Similarly, the requirement that workers stay in the client's house until \PC{} has received payment again suggests that little thought had been given to the workers' experiences of the system, and serves as an example of the workers themselves being the site of engagement between the client and a corporate entity: placing additional focus and pressure upon the worker's physical presence and emotional performance \cite{raval2016}. Finally, that \PC{}'s platform is built around technology that the client has access to but the worker does not---with the IVR component's existence being an attempt to navigate this core inequality---highlights that the system's human-centred design focus is on the clients, not the workers.
    
    \item[Loss of focus on workers] While \NGO{} made fewer contributions to the design, they were consistent in evidencing their stated goal of improving the well-being of women domestic workers: suggestions such as the algorithm accounting for workers' ratings of clients and the inclusion of safety reminders had a clear focus on improving the workers' safety and agency. However, within the final design the focus on women's empowerment was lost, and the women were no longer the priority. As the party with the platform and task of creating the implementation, \PC{} had the most control over the system's final design. Despite producing some concrete improvements towards the upholding of the standards set out in the DWPWP, as a for-profit company \PC{}'s priority was understandably to make money, and so the focus of the project naturally shifted from the women being the cause to being the product: no longer the priority, but a commodity in a capitalist process. 
    
\end{description}

We present these insights as evidence of how values levers can be applied by designers as a simple tool to reveal and reflect upon what values are embedded and prioritised within their designs and processes. As Martin et al. argue, technology interventions introduced without care into gig-economy ecosystems have the potential to reify existing inequalities or create new ones \cite{martin2016}. Before attempting to introduce what we think are worker-centred interventions \cite{carlos2021}, we as designers and engineers need to reflect on our own interests, and how they compare and contrast with those who will be affected by the systems we create. While actually engaging with the workers in consultation or even a full co-design process would naturally be a more effective way to ensure a more suitable human-centred final design \citep{maguire2001}, these practices are frequently overlooked within commercial settings \cite{Ardito2014, Shilton2013}. We argue that in such cases, \textit{some} level of critical values reflection is better than none, and that values levers are a useful and simple tool for engaging in this space without requiring additional investment or expertise.

\subsection{Using Constraints to Introduce Values Levers in Design Practice}

% While the use of artefacts as design tools (e.g. \cite{GrowAGame, Alshehri2020}) is widely valued by UX designers and researchers as a method of prompting critical and creative discussions between stakeholders, such reflective processes are often undesirable within pragmatic, production-focused development environments \cite{Shilton2013}. We argue that not only are constraints-based values levers useful tools for reflecting on the values and assumptions being embedded into systems, but that they can also be realistically applied in production-focused contexts: this study was undertaken in partnership with a private for-profit company, gaining insights from their ongoing development of live infrastructure. We propose that introducing constraints in system design exercises could be used by system designers and engineers as an efficient way of surfacing and reflecting upon what values they are prone to embed within their platforms and why it is important to do so: even in real-world, for-profit contexts.

% \subsubsection{Highlight the Value of Including Representatives from All Stakeholders}
% An obvious flaw of this project was the lack of consultation or co-design with the system's main stakeholders: the domestic workers themselves, whose empowerment was meant to be the project's primary goal.
Many of the values levers discussed above were resultant from decisions that \PC{} and \NGO{} were forced to make due to the limitations of working within the IVR medium. Within an IVR system, menus and messages are usually prescribed, linear, largely static and should be limited in length to support cognition \cite{Suhm2008}. Such limitations require judgements to be made not only on the inclusion of information, but in what order it is presented. Furthermore, most traditional IVR systems act as blunt instruments: the end-user is given little ability to work outside of the confines of the design space set out by its system designer. Due to these limitations, the choices of what information and functionality is excluded and how included elements are ordered are strong indications of the designers' priorities and assumptions about the end user.

Within this study, the constraints of IVR required \PC{} and \NGO{} to negotiate the prioritisation particular information (e.g. logistical information for getting to a job), and the exclusion of that deemed too complicated or not worth the additional menu space (e.g. details of what the job would actually entail, or how the client has been rated by other workers). These choices often involved balancing the empowerment of domestic workers against the designers' other interests. We argue that by forcing these decisions to take place, the process of encountering the constraints and limitations of the technology became a values lever. 

As a technology medium, IVR is particularly well-suited to this due to its inherent limitations that require designers to go through a process of explicit prioritisation. However, we also posit that such constraints are often naturally encountered during the practice of technology development. Choices in other mediums might include: which items should be surfaced in the top level of a GUI; the minimum requirements for a user's hardware or version of operating system; the granularity of users' control over collection of their data; or reliance on the user having reliable internet access. In this regard, any design materials which feature constraints can serve as an effective way of surfacing stakeholders' values by forcing them to make decisions. 

These decisions could also act as a useful way of communicating and highlighting the practical effects of designing towards a given agenda, supporting the inclusion of less technical stakeholders in the decision-making process. Some functionality, such as the design of system infrastructure or algorithms, can be so complex to lead to practical opacity, often impeding productive critical analysis and inviting users to speculate and sense-make through channels external to the platform \cite{lee2015}. In cases where complexity impedes open discussion, major design decisions could be clarified and represented through abstraction: previous work has shown the benefits of using mediums such as Lego as design tools for collaboration and the communication of complex ideas \cite{Cantoni2009}. Such use of artefacts as design tools (e.g. \cite{GrowAGame, Alshehri2020}) is widely valued by UX designers and researchers as a method of prompting critical and creative discussions between stakeholders. However, this study is an example of how traditional reflective processes are often neglected within pragmatic, production-focused development environments \cite{Shilton2013}, where design is often based on tacit knowledge and `\textit{watching trends and chasing after innovations}' without first laying a human-centred groundwork for design \citep{ogunyemi2016}. Such was explicit for \PC{}, who aspired to be `\textit{the Uber for domestic workers}' and frequently changed operational model. In small- to medium-sized development environments, it can be harder to justify allocating resources to practices beyond those perceived to be core to the development of technology. Constraints-based values levers are likely to be more appealing mechanisms for including and prompting reflection on stakeholder values, as they can be more tightly integrated into the practical stages of development.

We also propose that engaging with decision-based values levers can be of value even beyond the production more appropriate designs: keeping records of design decisions---and the explicit values-based reasoning behind them---would be useful for the purposes of accountability and evidencing deliverables. For example, in this project \NGO{} and \PC{} could have been able to show a paper-trail of design decisions, highlighting to the funding bodies of the \SRP{} project how the system promoted the values and enforcement of the DWPWP. Such practical benefits offer further grounds for the integration of reflection through values levers into software design and development, in ways that `pragmatic' practitioners may find more palatable.

However, this study also showed that the introduction of values levers is not a guarantee of worker-centred design: despite extensive dialogue between the parties, our findings showed the final design to still more closely reflect the values of \PC{} over \NGO{} or the research team. Examples of issues discussed within design sessions but (as-yet) unimplemented include our concerns relating to the application giving clients the workers' phone numbers, and \NGO{}'s suggestion that a client's rating by workers should be accounted for by the worker selection algorithm. While these topics were raised through the IVR design process as points for critical discussion, we argue that the fact that these issues and suggested improvements were not acted upon is a separate issue: likely resulting from \PC{} having significantly greater control over the design and implementation than \NGO{} or ourselves. In this case, the constraints acted as tools to reveal what the different stakeholder values were and how they were being prioritised and represented within the design: an opportunity to surface the implicit positions of the designers, even if not all that was surfaced was acted upon.


% \subsubsection{Mitigate Power Differentials Between Stakeholders}

% The application of values levers doesn't necessarily mean that a produced design will be worker-centred, however: as seen in this project, critical discussions and an awareness of potential issues may not always be enough to produce end designs which satisfy all parties. We posit that in this study, this was partly due to one party having the majority of the control over the design and implementation of the platform, accentuating the representation of their values and agendas within the final design over those of the other stakeholders. We therefore recommend that even when stakeholders are included in a design process, measures are taken during design engagements to not only ensure that that there is a continuing process of communication and accountability in situations of unequal distributions of power between stakeholders, requiring that all are kept abreast of developments and how issues are being addressed on a continuing basis by those with greater control over the design's implementation.