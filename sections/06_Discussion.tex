\section{Discussion}

These findings highlighted not only that elements of both \NGO{} and \PC{}'s values and assumptions were embedded into the final design of the system, but that process of designing the IVR provided an excellent entrypoint into discussions around these values. This section discusses what values and assumptions were surfaced, how the IVR acted as a values lever to enable these discussions, and provides recommendations for how constraints can be utilised as design tools for engaging with stakeholder values.

\subsection{Embedded Values in the System}

Despite \PC{} and \NGO{} believing that the IVR system would offer more objectivity than the `local guides', infrastructures can contain their own biases which are often informed by the values and priorities of those involved in the design process \citep{winner1980}. Likewise, analysing and querying the design of technological systems can give insights into the designer's priorities, attitudes and assumptions \citep{Star1999}. We argue that these findings highlight how the design decisions made during this process are indicative of each party's priorities and their perceptions of the domestic workers.

Firstly, there are multiple elements within the system's design and observations made during the design process which suggest that \PC{} had low expectations of the workers. For example, the `tracking calls' were perceived by \PC{} to be a necessary inclusion to prompt workers to arrive at client's houses on time---even after a high success rate during the `Human IVR' deployment (where the calls had been removed due to complaints from the workers), \PC{} was concerned that workers would not leave on time if the calls were not coming from a human operator. As in crowdsourcing platforms, this intrusive use of technology for monitoring frames workers as `troublesome components' to be controlled, rather than human stakeholders to be designed for \cite{martin2016, Irani2013}. \PC{} also believed that the women would be confused if given the ratings of clients when offered work, and that such decisions should be made algorithmically `\textit{from the back-end}'. Furthermore, while clients had access to workers' phone numbers, workers were only able to directly connect to clients once by routing through the IVR system. These low expectations and unequal provision of information and control contribute towards a final design where the worker has little power: as seen in analyses of other gig-economy systems \cite{martin2016, Hara2018, carlos2021, lee2015}, a lack of information flow and opaque, algorithmic control often results in a system where the worker has less power than other parties. 

Further inferences can be drawn regarding \PC's priorities---as noted, \PC{} are principally a for-profit company, and the system's design reflects their prioritisation of clients' interests over those of the workers. This is evident through design decisions both benign (e.g. the worker is called to accept a job before the client is called to confirm it is legitimate; that workers receive automated calls, while customers are called by human operators) and those that have serious potential consequences (e.g. that workers' ratings of clients were not provided during IVR calls offering jobs, nor being considered by the algorithm during `Human IVR' deployment; that \PC{} had few concerns related to safety when sharing workers' phone numbers with clients). As Lee argues, supply-demand orientated algorithmic controls frequently do not account for human factors \citep{lee2015}: such issues are further evident in this system, where workers are pushed through repeated phone calls during travel, prioritising keeping clients' waiting time to a minimum. The requirement that workers stay in the client's house until \PC{} has received payment again suggests that little thought had been given to the workers' experiences of the system, and serves as an example of the workers themselves being the site of engagement between the client and a corporate entity: placing additional focus and pressure upon the worker's physical presence and emotional performance \cite{raval2016}. Finally, that \PC{}'s platform is built around technology that the client has access to but the worker does not---with the IVR component's existence being an attempt to navigate this core inequality---highlights that the system's human-centred design focus is on the users, not the workers.

While \NGO{} made fewer contributions to the design, they were consistent in evidencing their stated goal of improving the well-being of women domestic workers. Suggestions such as the algorithm accounting for workers' ratings of clients and the inclusion of safety reminders had a clear focus on improving the workers' safety and agency. However, within the final design the focus on women's empowerment was lost, and the women were no longer the priority. As the party with the platform and task of creating the implementation, \PC{} had the most control over the system's final design. Despite producing some concrete improvements towards the upholding of the standards set out in the DWPWP, as a for-profit company \PC{}'s priority was understandably to make money, and so the focus of the project naturally shifted from the women being the cause to being the product: no longer the priority, but a commodity in a capitalist process. We argue that these findings reinforce the importance of designers being aware of each other's values, priorities and assumptions: as Martin et al. argue, technology interventions introduced without care into gig-economy ecosystems have the potential to reify existing inequalities or create new ones \cite{martin2016}. We also posit that the lack of consultation with the workers also meant that \PC{} and \NGO{}'s existing assumptions and values were more likely to be embedded in the design. Before attempting to introduce what we think are worker-centred interventions \cite{carlos2021}, we as designers and engineers need to reflect on our own interests, and how they compare and contrast with those who will be affected by the systems we create. Doing so requires an understanding of the values and interests of all involved.


\subsection{Technology Constraints as Values Levers}

As noted previously, IVR is a technology medium with inherent design limitations that require designers to go through a process of prioritisation. Menus and messages are usually prescribed, linear, largely static and should be limited in length to support cognition \cite{Suhm2008}, meaning that decisions be made about what information and functionality the user is given access to. Such limitations require judgements be made not only on the inclusion of information, but in what order it is presented. Furthermore, most traditional IVR systems act as blunt instruments, where the end-user is given little ability to work outside of the confines of the design space set out by its system designer. Due to these limitations, the choices of what information and functionality is excluded and how included elements are ordered are strong indications of the designers' priorities and assumptions about the end user. 

As such, the process of designing the IVR component itself became a values lever \cite{Shilton2013}. We argue that Shilton's given examples of limitations placed upon designers and engineers by platforms and institutions \cite{shilton2018, shilton2019} can be directly compared to the process of designing an IVR system: all of these sets of limitations acted as entry points for reflection and value discussions. In this study, we have been able to provide insights into how the stakeholders' values were embedded within the final system design, as direct results of questioning and critically analysing presented designs of the IVR platform: querying the inclusion, prioritisation, and omission of pieces of information and functionality accessible to workers through it, and how such decisions affect and reflect other aspects of the platform, such as the worker selection algorithm. As these discussions took place during the design process, the design itself came to be shaped by them and the values of those involved: examples include the inclusion of \NGO{}'s safety measures; \PC{} taking our feedback relating to prioritising quick access to emergency support over more commonly used menu options related to active jobs; and the algorithm's worker selection metrics, with \NGO{} and \PC{} negotiating a new solution that prioritised and including limits on worker distance to a mutually beneficial result.

However, this study also showed that the introduction of values levers is not a `magic bullet' to produce a value and worker-centred design: despite extensive dialogue between the parties, our findings showed the final design to still more closely reflect the values of \PC{} over \NGO{} or the research team. Examples of issues discussed within design sessions but as-yet unimplemented include: our concerns relating to the application giving clients the workers' phone numbers, and \NGO{}'s suggestion that a client's rating by workers should be accounted for by the worker selection algorithm. While these topics were raised through the IVR design process as points for critical discussion, we argue that the fact that these issues and suggested improvements were not acted upon is a separate issue: resulting from \PC{} having significantly greater control over the design and implementation than \NGO{} or ourselves. 

\subsection{Recommendations for Introducing Values Levers in Design Tools}

While the use of artefacts as design tools (e.g. \cite{GrowAGame, Alshehri2020}) is widely valued by UX designers and researchers as a method of prompting critical and creative discussions between stakeholders, such reflective processes are often undesirable within pragmatic, production-focused development environments \cite{Shilton2013}. We argue that not only are constraints-based values levers useful tools for reflecting on the values and assumptions being embedded into systems, but that they can also be realistically applied in production-focused contexts: this study was undertaken in partnership with a private for-profit company, gaining insights from their ongoing development of live infrastructure. We propose that introducing constraints in system design exercises could be used by system designers and engineers as an efficient way of surfacing and reflecting upon what values they are prone to embed within their platforms, even in real-world, for-profit contexts. Based on our insights gained from this study, we offer suggestions for how system design exercises could effectively configured to introduce values levers:

\subsubsection{Include Representatives from All Stakeholders}

An obvious flaw of this project was the lack of consultation or co-design with the system's main stakeholders: the domestic workers themselves, whose empowerment was meant to be the project's primary goal. Beyond advocacy by \NGO{}, the workers only became included in the design process during the late `Human IVR' stage, where their negative feedback relating to the tracking calls surprised \PC{}. Including more stakeholders' domain knowledge, values, and motives within the critical reflection process may be challenging due to increased odds of encountering tension points; however, as shown, these would likely have been encountered anyway by the introduction of the final design. Highlighting, navigating and negotiating through all stakeholders' values and assumptions earlier in the design process to find commonalities and confluences would likely have produced a more suitable final product. Furthermore, explaining design decisions across disciplinary barriers can serve as a form of values lever in itself \cite{shilton2018}, and including workers in the design of the platforms they work in could help reduce the technology's perceived opacity: preventing mistrust and reducing the impact of power asymmetry \cite{lee2015, martin2016, carlos2021}.


\subsubsection{Introduce Constraints to Require Explicit Decision-Making}
In this paper, we have argued that introducing design materials which feature resource-based constraints can serve as an effective way of surfacing stakeholders' values by forcing them to make decisions. The empowerment of domestic workers was shown to be one of multiple values which had to be negotiated alongside other (i.e. commercial) interests: a fact highlighted by the limitations of the IVR medium, which required stakeholders to prioritise particular information and functionality above others. While IVR was already conveniently constrained by its nature, new constraints can be contrived through abstraction if necessary: previous work has shown the benefits of using mediums such as Lego as design tools for collaboration and the communication of complex ideas \cite{Cantoni2009}. Such empirical techniques could assist in the discovery of unexpected values through defamiliarization \cite{LeDantec2009}, or be utilised to communicate the functionality of (and critique the values embedded into) the designs of system infrastructure such as algorithms: where their complexity frequently leads to practical opacity, impeding productive critical analysis and inviting speculation through external channels \cite{lee2015}.

\subsubsection{Mitigate Power Differentials Between Stakeholders}

The inclusion of stakeholders and the application of values levers doesn't necessarily mean that a produced design will be worker-centred, however: as seen in this project, critical discussions and an awareness of potential issues may not always be enough to produce end designs which satisfy all parties. We posit that in this study, this was partly due to one party having the majority of the control over the design and implementation of the platform, accentuating the representation of their values and agendas within the final design over those of the other stakeholders. We therefore recommend that measures are taken during design engagements to not only ensure that all stakeholder voices are heard, but that there is a continuing process of communication and accountability in situations of unequal distributions of power between stakeholders, requiring that all are kept abreast of developments and how issues are being addressed on a continuing basis by those with greater control over the design's implementation.