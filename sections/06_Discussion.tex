\section{Discussion}

\subsection{Embedded values in the system}
********* TODO
What this revealed about the companies perceptions of the women. Employees vs commodity.
Analysing and querying the design of infrastructure can give insights into the designer's attitudes and assumptions \citep{Star1999} ********* 

\subsection{Who was this designed for?}
*********  TODO 
The worker or the client? 
Throughout this process - the main beneficiary seems be the client, as is often seen with mainstream gig economy platforms. This is bc they are the ones paying the money. This is DESPITE Oxfam’s primary agenda being to empower women.. Focus on women was lost, by the end the women were not the priority. They became a pawn in the design process to support an organisation make money. ********* 

\subsection{Breaking the cycle of disempowerment}

\cite{Hara2018} and \cite{martin2016} both highlight the importance of technology in information flow within gig platforms, and how access to this information is strongly linked to power. such issues further exacerbated when the workers lack the same access to technology as the clients and platform holders

********* TODO What Oxfam was attempting to do was to work ‘within the gig economy system’.  Obfuscation - empowerment within a framework of disempowerment. Grey wash? There are many companies currently starting up similar programs in BD and they wanted to get in a use HT as a ‘gold star example’ of how an ethical platform could be used. But - this process revealed how working within the established paradigm wouldn’t shift the cycle of disempowerment these women constantly faced. e.g. (1) Socio-cultural perceptions embedded in the system. e.g. class - the distrust of these women, the intention to track and surveil, privacy breaches.. whilst there were continuous discussions of ‘empowerment’, the underlying values and ideals around what these women represented wasn’t addressed. In fact - these values were reflected in decisions that were embedded in the platform design. **************


\subsection{Platform cooperatives: an argument for working outside the system}

************* TODO Significant argument to work outside the system - disrupt. e.g. Platform cooperativism - creating platforms that reflect the values of the people. This becomes about who they are designing for. *********