\section{Discussion}

Analysing and querying the design of infrastructure can give insights into the designer's attitudes and assumptions

FROM THESIS ------------------
In her ethnographical study of physical infrastructure (such as the sewers and power supplies of cities, and the stairs and ramps of building entrances), Star argues that meaningful ethnographic study of these systems can open up an \textit{`ecological understanding'} of place \citep{Star1999}. She posits that infrastructure is \textit{`both relational and ecological---it means different things to different groups and is part of the balance of action, tools and the built environment, inseparable from them'}. In this regard, the humanist geographer's place and Star's infrastructure are similar in many ways. Star argues that the study of the physical infrastructure of sewerage, water and power supplies within cities can help one gain insights into distributional justice and planning power. One given example is that a keen observation of the usage (or omission) of stairs, ramps and railings can give an impression of institutional attitudes towards---and considerations of---people living with physical disabilities. A modern example of infrastructure reflecting an unequal distribution of power would be the water crisis in Flint, Michigan \citep{Clark2018}---the cause and response to which has been labelled as systemic environmental racism \citep{MichiganCivilRightsCommission2017a}. Star recalls that few participants in one of her projects utilised the final system that her team designed, despite the researchers following the principles of participatory design throughout the process. They identified that this was not because of usability issues with the interface, but rather how their design was a poor fit with the infrastructures the participants had to work with. The article highlights that the study of infrastructure in an ethnographic enquiry can uncover tacit conventions of everyday practices, allowing the unpacking of relationships between different communities, interest groups and perspectives.
--------------------------------