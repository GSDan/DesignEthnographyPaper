\section{Conclusion}

This paper has reported on the findings of a design ethnography undertaken with an international NGO and a for-profit gig-economy company in Bangladesh, during the design of an IVR component for use with a disadvantaged workforce of domestic workers. We argue that these design discussions regarding the IVR system acted as a `values lever': that the limited nature of the IVR format introduced a number of constraints which required the stakeholders to make design decisions which prioritised particular qualities within the system, and that discussing these decisions surfaced their values in relation to the project and their assumptions about the workers. We offer suggestions for how to introduce such values levers in technology production-focused contexts, arguing that the introduction of constraints could be a useful technique for values reflection: particularly in projects where parties may have different---even conflicting---agendas. As such a relationship is seemingly almost guaranteed between for-profit gig-economy companies and their employees, we argue that such steps should be considered during the production of any platforms looking to develop a worker-centred perspective.