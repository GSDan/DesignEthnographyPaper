\section{Limitations and Future Work}

The intended users of the system, the domestic workers, were not included in the design process. However, as noted, the practices of co-design and stakeholder consultation are often foregone in commercial development settings \cite{Ardito2014, Shilton2013}. We advocate for the inclusion of all stakeholders within participatory design practices---especially those who are most marginalised---but understand that this is rarely the case in practice.

This research project did not set out to evaluate constrained design through the lens of values levers: the constraints and the resulting discussions were identified as values levers after the study's conclusion. As such, the values levers we have identified in this paper were not explicitly engaged with as a tool for reflection by the stakeholders themselves within a formalised process. We aim to continue this research by exploring how constrained design can be utilised to incorporate values-based reflection into ongoing development processes, and how such processes can be designed to support accountability and transparency in contexts where stakeholders hold unequal levels of control and power.

\section{Conclusion}

This paper has reported on the findings of a design ethnography, undertaken with an international NGO and a for-profit gig-economy company in Bangladesh during the design of an IVR component for use with a disadvantaged workforce of domestic workers. We argue that the limited nature of the IVR format acted as a 'values lever': that it introduced a number of constraints which required the stakeholders to make design decisions which prioritised particular qualities within the system, and that discussing these decisions surfaced their values in relation to the project and their assumptions about the workers. We posit that such constraint-based design decisions are present in other technology mediums, and that as values levers they offer opportunities for engagement in critical, values-focused reflection: even in resource constrained development contexts. We argue that such reflection is particularly valuable in projects where parties may have different---even conflicting---agendas (such as between for-profit gig economy companies and their employees), and that constraints should be considered as values levers during the production of any platforms looking to develop a worker- or user-centred perspective.