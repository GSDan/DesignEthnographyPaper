\section{Methodology}

This study took place over the course of 11 months, performed remotely due to the global pandemic. During this time, the research team hosted 12 Zoom meetings with members of \PC{} (\PCOne{} \& \PCTwo{}), with each meeting running for an average of one hour. At least one representative from \NGO{} also attended eight of these meetings. As well as providing design suggestions, we approached these engagements through the lens of design ethnography [cite]: a qualitative research approach which, by being embedded in the design process, allowed us to gain deeper insights into the project's stakeholders' policies and practices by maintaining a critical lens on the design process and actively gathering data. These sessions were either recorded using Zoom's built-in recorder and transcribed, or a member of the research team took detailed notes during the meeting.

During the period of this study, \PC{} progressed the design through three stages. The first consisted of preliminary discussions, where we queried \PC{} and \NGO{} about the project to understand and advise on the design requirements and to gain an understanding of the company's services, business model and structure. The second was an iterative design stage, where \PCTwo{} produced several iterations of the IVR system design in Microsoft PowerPoint, detailing the `IVR flow': the menus, information and options available to the domestic workers as they interact with the system over the phone. These PowerPoint files were shared with us after each meeting. During the meetings, \PCTwo{} shared their screen to walk through the updated designs for discussion and feedback. The research team used these design decisions as probes for starting discussions around the challenges faced by the company and (less directly) their values, priorities and tensions with the project's other stakeholders. The final stage of this study saw \PC{} introduce an operational prototype, which completely supplanted the Local Guides while the actual IVR system was being built. Dubbed the `Human IVR', this acted like a `Wizard of Oz' technology prototype: in-lieu of an automated system, \PC{} used human call operators who followed the designed IVR script and the system's algorithmic recommendations to contact and interact with the domestic workers. Meetings during this stage focused on this prototype's implementation, how it was performing and what feedback \PC{} had received about it from the domestic workers.

To assess if the final platform design fulfilled \NGO{}'s project goals, the research team performed an analysis of the meeting notes, transcriptions and the system's designs to identify moments throughout which are pertinent to the research question `\textit{how did the varying agendas and values of the three stakeholders shape the experiences of the domestic workers they were seeking to empower?}' Given that the women's experiences would primarily be with the IVR system, this necessitated that the analysis focus on the women's direct and indirect interactions with it. The analysis consisted of two phases, with two different goals: i) to understand the tacit aims of the designed interaction; ii) to draw on the data to explore the underlying motivations behind it as a design decision. 