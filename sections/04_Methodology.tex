\section{Methodology}

This study took place over the course of 11 months, performed remotely due to the global pandemic. During this time, the research team hosted 12 Zoom meetings with representatives of \PC{}, with each meeting running for an average of one hour and all parties providing design suggestions. At least one representative from \NGO{} also attended eight of these meetings. These sessions were either recorded using Zoom's built-in recorder and transcribed, or a member of the research team took detailed notes during the meeting. Framed through discussions regarding the design of the IVR component, we approached this study through the lens of design ethnography: a qualitative research approach which, by being embedded in the design process, allowed us to gain deeper insights into the project's stakeholders' policies and practices by maintaining a critical lens on the design process.

During the period of this study, \PC{} progressed the design process through three stages: 

\begin{enumerate}
\item Preliminary discussions, where we queried \PC{} and \NGO{} about the project to gain an understanding of the company's services, business model, structure and challenges.  

\item Iterative design, where \PC{} produced several iterations of the IVR system design in Microsoft PowerPoint, detailing the `IVR flow': the menus, information and options available to the domestic workers as they interact with the system over the phone. These PowerPoint files were shared with us after each meeting. During the meetings, \PC{} shared their screen to walk through the updated designs for discussion and feedback. The research team used these designs as probes for starting discussions around design decisions made by \PC{} and \NGO{}.

\item An operational prototype, which completely supplanted the local guides while the actual IVR system was being built. Dubbed the `Human IVR', this acted like a `Wizard of Oz' technology prototype: in-lieu of an automated system, \PC{} used human call operators who followed the system's algorithmic recommendations and the designed IVR script to contact and interact with the domestic workers. Meetings during this stage focused on this prototype's implementation, how it was performing, and what feedback \PC{} had received about it from the domestic workers.
\end{enumerate}

To assess if the final platform design fulfilled \NGO{}'s project goals, the research team reviewed the meeting notes, transcriptions and the system's designs to identify moments throughout which are pertinent to the research question `\textit{how did the varying agendas and values of the three stakeholders shape the experiences of the domestic workers they were seeking to empower?}' Given that the women's experiences would primarily be with the IVR system, this necessitated that the review focus on the direct and indirect interactions that they would have with it: identifying each interaction's function, and understanding the underlying motivations behind it as a design decision. Note that the domestic workers themselves were not consulted by \PC{} or \NGO{} during this process, and first encountered the design during the testing of the `Human IVR'. While we as researchers are advocates of participatory and co-design design methodologies, our role was to observe, query and provide design feedback during development.